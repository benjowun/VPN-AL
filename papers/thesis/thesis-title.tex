% --- English Title Page ------------------------------------------------
\begin{titlepage}
\begin{center}
\begin{figure}[!ht]
\centerline{\includegraphics[width=4cm,keepaspectratio=true]{images/tug}}
\end{figure}	
	
\vspace*{28mm}

{\LARGE Benjamin Wunderling\textsuperscript{1}, BSc}\\

\vspace{16mm}

{\LARGE \bf  Model Learning and Fuzzing of the \\ IPsec-IKEv1 VPN Protocol\\}

\vspace{16mm}

{\Large \bf MASTER'S THESIS}\\

\vspace{6mm}

{\large to achieve the university degree of}

\vspace{4mm}

{\large Diplom-Ingenieur}

\vspace{4mm}

{\large Master's degree programme: Computer Science}

\vspace{16mm}

{\large submitted to}

\vspace{4mm}

{\large \bf Graz University of Technology}

\vspace{12mm}

{\large Main Supervisor}

{\large Ao.Univ.-Prof.\ Dipl.-Ing.\ Dr.techn.\ Bernhard\ K.\ Aichernig}

{\large Institute of Software Technology (IST)}

\vspace{6mm}

{\large Co-Supervisor}

{\large Dipl.-Ing.\ Andrea\ Pferscher \ BSc}

{\large Institute of Software Technology (IST)}
\vspace{4mm}


\vfill
{Graz, 05. September 2023}
%
%{\Large
%Institute of Software Technology (IST)\\
%\\
%A-8010 Graz, Austria\\}
%
%
%
%\begin{figure}[!ht]
%\centerline{\includegraphics[width=4cm,keepaspectratio=true]{fig/tug}}
%\end{figure}
%
%\vspace{16mm}
%
%{\large
%\begin{tabular}{ll}
%Advisor:    & Ao.Univ.-Prof.\ Dipl.-Ing.\ Dr.techn.\ Bernhard K. Aichernig\\[0.5ex]
%
%\end{tabular}}
%
%\vfill
%{\large Graz, 16. Mai 2019}
%\vspace{15mm}
\end{center}

\noindent
\underline{\hspace*{3cm}}\\
{\footnotesize
\textsuperscript{1} E-mail: benjamin.wunderling@gmail.com\\
\copyright ~ Copyright 2023 by the author}

\end{titlepage}
% --- Pledge ------------------------------------------------------------
\newpage
\shipout\null
\vspace*{20mm}

\begin{center}
	{\Large\bfseries AFFIDAVIT}
\end{center}
\vspace{5mm}
\noindent
I declare that I have authored this thesis independently, that I have not used other than  the  declared  sources/resources,  and  that  I  have  explicitly  indicated  all  material which has been quoted either literally or by content from the sources used. The text document uploaded to TUGRAZonline is identical to the present master's thesis.

\vspace{2cm}

\noindent
\begin{tabular}{ccc}
	\hspace*{6cm}     & \hspace*{2cm}   & \hspace*{6.7cm}\\
	\dotfill          &                 & \dotfill\\
		   Date       &                 & Signature\\
\end{tabular}

\vspace{35mm}


% --- English Abstract --------------------------------------------------
\pagestyle{plain}
\pagenumbering{roman}
\newpage

\vspace*{25mm}

\begin{changemargin}{15mm}{15mm}
\begin{center}
{\Large\bfseries Abstract}
\end{center}
\vspace*{7mm}

In light of the growing number of individuals working from home, the importance of secure \acl{vpn} communication protocols has increased significantly. Consequently, the need for rigorous security testing of \acl{vpn} software has become more critical than ever before. 
In this thesis, we employ \acl{aal} to demonstrate how behavioral models of two black-box \acl{ipsec} \acl{vpn} implementations can be automatically learned. We subsequently utilize the learned models to facilitate security testing the \acl{ipsec} implementations through model-based fuzzing. 
More specifically, we first learn behavioral models of strongSwan and libreswan \acl{ipsec} servers using both the $L^*$ and $KV$ model learning algorithms. Learned models of both \acl{ipsec} implementations are presented and the utilized learning algorithms are compared and evaluated. The learned models are then used to perform model-based fuzzing of the respective implementations. While fuzzing, new behavior is detected by utilizing the previously-learned models as a reference. Additionally, we employ search-based methods, including a genetic algorithm-based approach, aided by the learned models, in order to enhance the effectiveness of fuzzing. Our fuzzing analysis uncovered several new security issues, including several standard violations and a potential deadlock state. Furthermore, our model learning approach led to the discovery of a bug in a cryptographic Python library.

 
\vspace{5mm}
\noindent
{\large\bfseries Keywords: IPsec $\cdot$ VPN $\cdot$ Active automata learning $\cdot$ Model-based fuzzing}




% --- German Abstract ---------------------------------------------------
\selectlanguage{austrian}
\newpage

\vspace*{10mm}

\begin{center}
{\Large\bfseries Kurzfassung}
\end{center}
\vspace*{2mm}

Aufgrund der stetig steigenden Anzahl an Personen, die von zu Hause aus arbeiten, hat die Wichtigkeit sicherer \acl{vpn}-Kommunikationsprotokolle erheblich zugenommen. Infolgedessen ist die Notwendigkeit strenger Sicherheitskontrollen für \acl{vpn}-Software wichtiger denn je zuvor. In dieser Masterarbeit verwenden wir \acl{aal}, um Verhaltensmodelle von zwei \acl{ipsec} \acl{vpn}-Implementationen, strongSwan und libreswan, automatisiert zu lernen. Dazu verwenden wir die $L^*$ und $KV$ Automatenlern-Algorithmen. Die Verhaltensmodelle und die fürs Lernen verwendeten Automatenlern-Algorithmen werden präsentiert und verglichen. Anschließend werden die erlernten Modelle verwendet um modell-basiertes Fuzzing der jeweiligen Implementierungen zu betreiben. Während dem Fuzzing wird neues Verhalten mithilfe der zuvor erlernten Modelle entdeckt. Des Weiteren setzen wir suchbasierte Methoden ein, inklusive eines genetischen Algorithmus, um die Effektivität des Fuzzings zu verbessern. Unsere Tests haben einige potenzielle Sicherheitslücken enthüllt, darunter mehrere Verstöße gegen Standards und ein potenzieller Deadlock-Zustand. Darüber hinaus hat unser Automatenlernen zu der Entdeckung eines Fehlers in einer kryptografischen Python-Library geführt.

\vspace{5mm}
\noindent
{\large\bfseries Schlagworte: IPsec $\cdot$ VPN $\cdot$ Aktives Automatenlernen $\cdot$ Modell-basiertes Fuzzing}


\selectlanguage{english}
\end{changemargin}

