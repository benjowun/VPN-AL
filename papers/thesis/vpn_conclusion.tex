%----------------------------------------------------------------
%
%  File    :  vpn_conclusion.tex
%
%  Author  :  Keith Andrews, IICM, TU Graz, Austria
% 
%  Created :  22 Feb 96
% 
%  Changed :  19 Feb 2004
% 
%----------------------------------------------------------------

\chapter{Conclusion}

\label{chap:Conclusion}

\TODO{improve on this}
\subsection{Conclusion}
In this Section, we showcased the four most relevant learned models and compared them with respect to various metrics including runtime and number of sent queries. Our results shows that our learning setup succeeds in its goal of reliably learning models of the target \ac{ipsec} \ac{ike}v1 server. We contrasted two popular model learning algorithms $KV$ and $L^*$ and explained why we consider $KV$ to be better suited for our learning setup. Additionally, we demonstrated the usefulness of \ac{aal} from a testing standpoint by showcasing a crypto library bug found during model learning. Future work could include support for more authentication methods, additional extensions and variable user-cookies. Additionally, it would be interesting to test the model-learning framework with multiple \ac{ipsec} implementations.

\subsection{Conclusion}
In this section, we presented the results of our fuzzer, highlighting the most interesting findings, most notably, two deviations from the RFC specification.  We recommend, that the findings be thoroughly examined, to ensure that they were not created accidentally and do not pose compatibility or security risks. Furthermore, we compared the two different utilized input sequence generation methods - filtering-bas ed and mutation-based. We found that both methods discovered the same findings, indicating their comparable effectiveness in generating and testing data. Therefore, we argue that the methods should be evaluated mainly based on their runtime performance. Comparing the two methods, we found that the mutation-based input sequence generation method performed significantly better, completing the fuzzing roughly 24 times faster than its counterpart (for a 60 mutation deep input sequence generation). This suggests, that the mutation-based method is better suited for larger-scale tests, where an exhaustive test might be too time costly. 
Overall, our findings highlight the importance of thorough testing and validation of network protocols and their implementations and show how new tools and techniques can be used to help accomplish that.

