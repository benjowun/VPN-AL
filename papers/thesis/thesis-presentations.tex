%----------------------------------------------------------------
%
%  File    :  thesis-presentations.tex
%
%  Author  :  Keith Andrews, ISDS, TU Graz, Austria
% 
%  Created :  03 Jan 2019
% 
%  Changed :  27 Mar 2019
% 
%----------------------------------------------------------------


\chapter{Giving a Presentation}

\label{chap:Presentation}


% \chapquote{
% Imitation is the sincerest flattery.
% }
% {
% Charles Caleb Colton, English writer, 1780--1832.
% }


Academic work is almost always presented in a talk or presentation at
some point in time. Giving a good presentation requires a careful
balance between spoken and visual material.


\section{Types of Presentation}

\textcite{SpeakingPowerPoint} distinguishes between four kinds of
presentation, depending on the size of the audience and the amount of
interaction between speaker and members of the audience:
\begin{itemize}
\item \liintro{Ballroom presentations}. \\
   Presented to a large audience,
  often in a darkened room. The speaker does all the talking (often no
  questions are allowed at the end), uses compelling visuals, and aims
  to entertain as well as inform.

\item \liintro{Briefing presentations}. \\
   Used in boardroom settings to
  perhaps one or two dozen people. The speaker does most of the
  talking, but some interaction is allowed.

\item \liintro{Discussion presentations}. \\
  Used for smaller groups upto
  say 10 people. The speaker does most of the talking at first, but
  discussion is then opened up.

\item \liintro{Reading presentations}. \\
  A slide deck read individually
  either on screen or paper. It must stand on its own, without spoken
  support.
\end{itemize}





\section{Guidelines for Presentations}

Doumont \parencite{ThreeLaws,cognitivestyle,Doumont-TreesMapsTheorems}
established four rules for professional communication:
\begin{enumerate}
\item[0.] \liintro{Define your purpose}. \\
  Define the message to be conveyed.

\item[1.] \liintro{Adapt to your audience}. \\
  Optimise the communication to the target audience.

\item[2.] \liintro{Maximise the signal-to-noise ratio}. \\
  Reduce or eliminate any extraneous ``noise'' which might
  distract from the message. Suppress rather than add.
  Remove every unnecessary drop of ink.

\item[3.] \liintro{Use effective redundancy}. \\
  Both the slide deck and spoken text should stand for themselves.  
  Text and visuals should reinforce each other: state the main
  point of a slide in concise text, reinforced visually
  as far as possible.
\end{enumerate}
The slides in a presentation should convey the main message, focusing
not on providing every detail, but rather on the implications that
follow from them. \textcite{Alley-CraftScientificPresentations-2Ed} is
another good guide to creating and giving scientific presentations.





\section{Guidelines for Effective Slides}


\subsection{Usability}

For usable slides:
\begin{itemize}
\item Slide layout, font sizes, and image placement should be
  consistent.

\item Fonts must be sufficiently large (readable at the back of the
  room).

\item The slide number (4 of 23) should be included at the bottom
  right of each slide.

\item In general, dark text on a light background is more readable,
  unless the room is completely dark.\\
  \shortnote{In a ballroom setting in a darkened room, light text
    on a dark background can be effective.}
\end{itemize}



\subsection{Minimise Distractions}

Effective slides should not compete for attention with the speaker:
\begin{itemize}
\item Use at most two typefaces, at few different sizes.

\item Use colour variations sparingly.

\item Eliminate purely decorative graphics or clip art.

\item Avoid flashy distracting backgrounds.
\end{itemize}



\subsection{Slide Content}

In terms of slide content:
\begin{itemize}
\item Carefully design slide headings.

\item Do not write full sentences. Reduce the number of words by
  clever rephrasing not random truncation. Bullet items should occupy
  at most two lines of text.

\item Where detailed tables, charts, or graphics would be helpful to
  convey the message, distribute them to audience members in
  the form of a handout.
\end{itemize}





\subsection{Academic Criteria}

For academic presentations, it is important to attribute textual
quotations and to state both attribution and permission for any
images used.

Instead of having one or more slides of references at the end of the
presentation:
\begin{itemize}
\item If you include a result or quotation from somewhere else, state
  the source as a footnote at the bottom right of the slide.  Link to
  the original, if possible.

\item If you include an image or a diagram from somewhere else, state
  both the source and permission as a footnote at the bottom right of
  the slide. Link to the original, if possible.
\end{itemize}

