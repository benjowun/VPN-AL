%----------------------------------------------------------------
%
%  File    :  thesis.tex
%
%  Author  :  Keith Andrews, ISDS, TU Graz, Austria
%
%  Created :  30 Jul 1997
% 
%  Changed :  22 Jan 2021
%
%----------------------------------------------------------------

\documentclass[11pt]{book}

\usepackage[
a4paper,
twoside,
top=5mm,                % top margin
bottom=7mm,             % bottom margin
inner=20mm,             % inner margin (next to binding)
outer=20mm,             % outer margin (opposite binding)
bindingoffset=10mm,     % on binding side
includeheadfoot,        % include head(er) and foot(er)
headheight=10mm,        % height of header
headsep=15mm,           % sep between header and text body
footskip=15mm,          % sep between body and baseline of footer
footnotesep = 10mm plus 2mm minus 0mm  % bottom of body to top of footnote
]{geometry}
% A4 paper is w=210m, h=297mm


\newcommand{\thisdate}{10 Nov 2021}  % date of this version
\newcommand{\thisyear}{2021}         % year of this version


\newcommand{\fullh}{24cm}         % height of figures for 1 per page
\newcommand{\halfh}{9.5cm}        % height of figures for 2 per page
\newcommand{\thirdh}{6cm}         % height of figures for 3 per page


\setlength{\parindent}{1em}       % less indentation
\setlength{\parskip}{1.5ex plus 0.3ex minus 0.3ex}  % vert. space before para


% \tolerance is set by LaTeX to 200
% \sloppy sets \tolerance = 9999
% which allows LaTeX more tolerance in adding word spacing

% \sloppy
% \fussy
% \tolerance = 1000

\tolerance=400 
% makes some lines with lots of white space, but      
% tends to prevent words from sticking out in the margin



\setcounter{secnumdepth}{3}     % lowest section level still numbered
\setcounter{tocdepth}{2}        % lowest section level entered in ToC

\usepackage[T1]{fontenc}        % 8-bit output chars (must be before inputenx)
\usepackage[utf8]{inputenx}     % input char encoding

\usepackage[english,austrian,british]{babel}

\usepackage{newtxtext}          % newer times fonts
\usepackage{newtxmath}

\usepackage{relsize}            % relative font sizes \smaller \larger
\usepackage{float}              % H for float placement
\usepackage{setspace}           % adjust line spacing

\usepackage{textcomp}           % symbols such as \texttimes and \texteuro
\usepackage{latexsym}
\usepackage{fontawesome}        % fontawesome symbols

\usepackage{siunitx}            % prettier number formatting
\sisetup{%
	group-separator={,},
	group-minimum-digits=4,
}
\usepackage[super]{nth}         % 1st, 2nd, 3rd, etc.

\usepackage{xspace}
\usepackage{xstring}            % string manipulation macros
\usepackage{xparse}             % commands with optional arguments
\usepackage{etoolbox}           % for \newrobustcmd
\usepackage{makecmds}           % for \makecommand
\usepackage{calc}               % for math calculations


\usepackage[svgnames,table,xcdraw]{xcolor}
\definecolor{darkgreen}{rgb}{0.0,0.2,0.0}
\definecolor{darkblue}{rgb}{0.0,0.0,0.2}
\definecolor{darkred}{rgb}{0.2,0.0,0.0}
\definecolor{verylightgrey}{gray}{0.95}
\definecolor{lightgrey}{gray}{0.9}
\definecolor{grey}{gray}{0.7}
\definecolor{black}{gray}{0.0}

\definecolor{tableheadercolour}{gray}{0.8}
\definecolor{tablerowcolour}{gray}{0.93}

\usepackage{longtable}
\usepackage{multirow}
\usepackage{tabularx}

% Define some new column types for tables:
% like X but flushleft (= raggedright) rather than justified
\newcolumntype{Y}{>{\raggedright\arraybackslash}X}
% a p column but flushleft (= raggedright) rather than justified
\newcolumntype{L}[1]{>{\raggedright\arraybackslash}p{#1}}
% a p column but flushright (= raggedleft) rather than justified
\newcolumntype{R}[1]{>{\raggedleft\arraybackslash}p{#1}}


\usepackage{booktabs}           % nicer tables

\newcommand{\tablestretch}
{\renewcommand{\arraystretch}{1.20}}  % spacing between table rows




\usepackage{verbdef}            % define robust verb strings
\usepackage{verbatim}
\usepackage{comment}



% better lists
\usepackage{enumitem}

\setlist{
	topsep=0pt,
	partopsep=0pt,
	parsep=0.6ex,
	itemsep=1.2ex,
	left=\parindent .. 2\parindent,    % bullet .. start ot text
}

\setlist[description]{
	style=sameline,
}

% acronyms
\usepackage[nolist,nohyperlinks]{acronym}
\begin{acronym}
	\acro{vpn}[VPN]{Virtual Private Network}
	\acro{ipsec}[IPsec]{Internet Protocol Security}
	\acro{ike}[IKE]{Internet Key Exchange protocol}
	\acro{sul}[SUL]{system under learning}
	\acro{sut}[SUT]{system under test}
	\acro{aal}[AAL]{active automata learning}
	\acro{mat}[MAT]{minimally adequate teacher}
	\acro{ah}[AH]{Authentication Header}
	\acro{esp}[ESP]{Encapsulating Security Payload}
	\acro{isakmp}[ISAKMP]{Internet Security Association and Key Management Protocol}
	\acro{sa}[SA]{Security Association}
	\acro{vm}[VM]{virtual machine}
	\acro{afl}[AFL]{american fuzzy lop}
	\acro{dfa}[DFA]{deterministic finite automaton}
	\acroplural{dfa}[DFA]{deterministic finite automata}
	\acro{mid}[M-ID]{message-ID}
	\acro{iv}[IV]{initialization vector}
	\acro{psk}[PSK]{pre-shared key}
	\acro{dpd}[DPD]{Dead Peer Detection}
	\acro{pfs}[PFS]{Perfect Forward Secrecy}
	\acro{oss}[OSS]{open source software}
\end{acronym}


% use caption and subfig (caption2 and subfigure are now obsolete)

\usepackage[
position=bottom,
margin=1cm,
font=small,
labelfont={bf,sf},
format=plain,
indention=5mm,
aboveskip=4mm,
belowskip=0mm,
]{caption,subfig}

\captionsetup[subfigure]{
	margin=0pt,
	parskip=0pt,
	indention=5mm,
	farskip=4mm,            % skip above subfig (assuming captions at bottom)
	captionskip=2mm,        % skip between subfig and subcaption
}



\usepackage{listings}                 % for listings of source code

\makeatletter
\newlength{\numwidth}%
\setlength{\numwidth}{\widthof{\normalfont{\lst@numberstyle{99}}}}% Up to 2-digit (99) line numbers
\def\lst@PlaceNumber{%
	\makebox[\numwidth+1em][l]{%
		\makebox[\numwidth][r]{\normalfont\lst@numberstyle{\thelstnumber}}%
	}%
}
\makeatother

% lstset strategy: define defaults here for
% all non-floating (displayed) listings
% floated listings override these settings later

\lstset{                              % set parameters for listings
	floatplacement=tp,                  % default float placement
	numberbychapter,
	inputencoding=utf8,
	language=,                          % empty = plain text
	tabsize=2,
	xleftmargin=2\parindent,
	xrightmargin=2\parindent,
	frame=none,
	framexleftmargin=0mm,
	rulesepcolor=\color{verylightgrey},
	numbers=left,
	stepnumber=1,
	numberstyle=\scriptsize,
	numbersep=2ex,
	breaklines,
	showtabs=false,
	showspaces=false,
	showstringspaces=false,
	%
	basicstyle=\small\ttfamily,
	keywordstyle=\color{black},
	identifierstyle=\color{black},
	commentstyle=\color{SteelBlue},
	stringstyle=\color{black},
	%
	captionpos=b,
	abovecaptionskip=\abovecaptionskip,
	belowcaptionskip=\belowcaptionskip,
	extendedchars=true,           % listings usually only support 7-bit ascii chars
	literate=%                    % map some one-byte utf8 chars for use in listings
	%    { }{{~}}1                   % non-breaking space
	{©}{{\textcopyright}}1
	{€}{{\texteuro}}1
	{Ö}{{\"O}}1
	{Ä}{{\"A}}1
	{Ü}{{\"U}}1
	{ß}{{\ss}}1
	{ö}{{\"o}}1
	{ä}{{\"a}}1
	{ü}{{\"u}}1,
}


\lstdefinelanguage{biblatex}   % based on biblatex v 2.7a from 2013-07-14
{
	keywords={%
		@article,@book,@mvbook,@inbook,@bookinbook,@suppbook,%
		@booklet,@collection,@mvcollection,@incollection,@suppcollection,%
		@manual,@misc,@online,@patent,@periodical,@suppperiodical,%
		@proceedings,@mvproceedings,@inproceedings,@reference,@mvreference,%
		@inreference,@report,@set,@thesis,@unpublished,@xdata,%
		@conference,@electronic,@mastersthesis,@phdthesis,@techreport,@www,%
		@artwork,@audio,@bibnote,@commentary,@image,@jurisdiction,@legislation,%
		@legal,@letter,@movie,@music,@performance,@review,@software,%
		@standard,@video%
	},
	sensitive=false,
	comment=[l][\itshape]{@comment},
	morecomment=[l]{\%},
}

\lstdefinelanguage{CSS}
{
	alsoletter={-},
	morekeywords={%
		color,background,background-color,margin,padding,font,
		font-family,weight,%
		display,position,top,left,right,bottom,list,%
		style,border,size,white,space,min,width%
	},
	sensitive=false,
	morecomment=[l]{//},
	morecomment=[s]{/*}{*/},
	morestring=[b]",
}

% very simple highlighting for listings
% only highlight comments and strings

\lstdefinelanguage{SVG}
{
	morestring=[b]",
	morecomment=[s]{<!--}{-->},
	commentstyle=\color{DarkOliveGreen},
	stringstyle=\color{Navy},
	identifierstyle=\color{black},
	keywordstyle=\color{black},
}


\lstdefinelanguage{TypeScript}
{
	keywords= {break, case, catch, continue, debugger, default, delete, do, else,
		finally, for, function, if, in, instanceof, new, return, switch,
		this, throw, try, typeof, var, void, while, with, await, async,
		case, catch, class, const, default, do, enum, export, extends,
		finally, from, implements, import, instanceof, let, static, super,
		switch, throw, try},
	morecomment=[l]{//},
	morecomment=[s]{/*}{*/},
	morestring=[b]',
	morestring=[b]",
	morestring=[b]`, % Interpolation strings.
	sensitive=true,
	commentstyle=\color{DarkOliveGreen},
	stringstyle=\color{Navy},
	identifierstyle=\color{black},
	keywordstyle=\color{black},
}






\usepackage[compact,nobottomtitles,pagestyles,explicit]{titlesec}
% when using explicit, must explicitly include #1 for titlename

% nobottomtitles
% move section headings close to page bottom to next page
\renewcommand{\bottomtitlespace}{2cm}

% \chaptermark sets the value of \chaptertitle for later
% \@chapapp is defined as \chaptername outside the appendix,
% and as \appendixname within the appendix.
\makeatletter
\titleformat{\chapter}
[display]                                            % shape
{\chaptermark{\thechapter~~#1}\sffamily\bfseries}    % format
{\huge\@chapapp\ \thechapter}                        % label
{4ex}                                                % sep
{\Huge#1}                                            % before-code
\makeatother

\titleformat{name=\chapter,numberless}
[block]                                              % shape
{\chaptermark{#1}\sffamily\bfseries}                 % format
{}                                                   % label
{0ex}                                                % sep
{\Huge#1}                                            % before-code

\titleformat{\section}
{\normalfont\Large\sffamily\bfseries}{\thesection}{0.8em}{#1}

\titleformat{\subsection}
{\normalfont\large\sffamily\bfseries}{\thesubsection}{0.8em}{#1}

\titleformat{\subsubsection}
{\normalfont\normalsize\sffamily\bfseries}{\thesubsubsection}{0.8em}{#1}

\titleformat{\paragraph}[runin]
{\normalfont\normalsize\sffamily\bfseries}{\theparagraph}{0.8em}{#1}

\titleformat{\subparagraph}[runin]
{\normalfont\normalsize\sffamily\bfseries}{\thesubparagraph}{0.8em}{#1}


% vertical spacing before and after section titles
\titlespacing*{\section}
{0pt}{3.5ex plus 0.5ex minus 0.5ex}{0ex plus 0ex minus 0.2ex}

\titlespacing*{\subsection}
{0pt}{2.5ex plus 0.5ex minus 0.5ex}{0ex plus 0ex minus 0.2ex}

\titlespacing*{\subsubsection}
{0pt}{2ex plus 0.5ex minus 0.5ex}{0ex plus 0ex minus 0.2ex}

\titlespacing*{\paragraph}
{0pt}{1.5ex plus 0.3ex minus 0.3ex}{0ex plus 0ex minus 0.15ex}

\titlespacing*{\subparagraph}
{0pt}{1ex plus 0.2ex minus 0.2ex}{0ex plus 0ex minus 0.1ex}


% define page headings how I want them

\newpagestyle{main}[\small]{
	% \addtolength\headheight{6.7pt}
	% \headrule
	\sethead%
	[{\parbox[t]{0.3\textwidth}%                    % even left
		{\sffamily\thepage}}]
	[]%                                             % even centre
	[{\parbox[t]{0.6\textwidth}%                    % even right
		{\raggedleft\sffamily\chaptertitle}}]
	{{\parbox[t]{0.6\textwidth}%                    % odd left
			{\sffamily\sectiontitle}}}%
	{}%                                             % odd centre
	{{\parbox[t]{0.3\textwidth}%                    % odd right
			{\raggedleft\sffamily\thepage}}}
}





\usepackage{titletoc}

% \contentsmargin{2.55em}

\titlecontents{chapter}%
[1.5em]%                         % left indent to entry text
{\addvspace{1em}\bfseries}%      % above-code per entry
{\contentslabel{1.5em}}%         % format for numbered entry
{\hspace*{-1.5em}}%              % format for unnumbered entry
{\hfill\contentspage}%           % [no dots] and page num per entry


% Note: \dottedcontents is short form of \titlecontents

\dottedcontents{section}%
[3.8em]%                         % left indent to entry text = 1.5 + 2.3
{}%                              % above-code per entry
{2.3em}%                         % label width
{1pc}%                           % space around the dots

\dottedcontents{subsection}%
[7.4em]%                         % left indent to entry text = 3.8 + 3.6
{}%                              % above-code per entry
{3.6em}%                         % label width
{1pc}%                           % space around the dots


\dottedcontents{figure}%         % LoF entries
[3.0em]%                         % left indent to entry text = 3.8 + 3.6
{}%                              % above-code per entry
{3.0em}%                         % label width
{1pc}%                           % space around the dots

\dottedcontents{table}%          % LoT entries
[3.0em]%                         % left indent to entry text = 3.8 + 3.6
{}%                              % above-code per entry
{3.0em}%                         % label width
{1pc}%                           % space around the dots



% List of Listings is unknown to titletoc, define here

% Add extra per-chapter space to LoL to mimic LoF and LoT
% (requires package etoolbox)
\makeatletter
\patchcmd{\@chapter}% <cmd>
{\addtocontents}% <search>
{\addtocontents{lol}{\protect\addvspace{10\p@}}% add per-chapter space
	\addtocontents}% <replace>
{}{}% <success><failure>
\makeatother

% Configure LoL to mimic LoF and LoT
\contentsuse{lstlisting}{lol}

\titlecontents{lstlisting}%
[3.0em]%                              % left indent
{\addvspace{1.5mm}}%                  % above-code per entry
{\contentslabel{3.0em}}%              % format for numbered entry
{\hspace*{-3.0em}}%                   % format for unnumbered entry
{\titlerule*[1pc]{.} \contentspage}%  % dots and page num per entry
[]%                                   % below-code per entry

\renewcommand{\lstlistlistingname}{List of Listings}






% sensible settings for floats

\setlength{\textfloatsep}{9mm plus 2mm minus 2mm}
\setlength{\floatsep}{9mm plus 2mm minus 2mm}
\setlength{\intextsep}{9mm plus 2mm minus 2mm}

\setlength{\dbltextfloatsep}{9mm plus 2mm minus 2mm}
\setlength{\dblfloatsep}{9mm plus 2mm minus 2mm}

\setlength{\abovecaptionskip}{4mm plus 2mm minus 1mm}
\setlength{\belowcaptionskip}{2mm plus 1mm minus 1mm}

% See http://www-rohan.sdsu.edu/~aty/bibliog/latex/floats.html
% See https://robjhyndman.com/hyndsight/latex-floats/

\setcounter{topnumber}{2}               % max num floats at top of page
\setcounter{dbltopnumber}{2}            % max num floats on 2col page
\setcounter{bottomnumber}{2}            % max num floats at bottom of page
\setcounter{totalnumber}{4}             % max num floats on a page

\renewcommand{\topfraction}{0.8}        % max fraction of floats at top
\renewcommand{\dbltopfraction}{0.9}     % max fraction of floats at top 2col
\renewcommand{\bottomfraction}{0.8}     % max fraction of floats at bottom
\renewcommand{\textfraction}{0.2}       % min fraction of text

% only for entirely float pages:
\renewcommand{\floatpagefraction}{0.7}      % min page fraction having floats
\renewcommand{\dblfloatpagefraction}{0.7}   % min 2col page fraction having floats


\usepackage[section,above,below]{placeins}  % keep floats to their own section





\usepackage[short]{datetime}   % load datetime *after* babel, requires fmtcount
% \newdateformat{britdate}{%
	% \ordinaldate{\THEDAY} \,\monthname[\THEMONTH] \THEYEAR
	% }
\newdateformat{unixdate}{%
	\twodigit{\THEDAY}~\shortmonthname[\THEMONTH]~\THEYEAR
}

% TODO: use new datetime2 instead of datetime



\usepackage[
autostyle=true,          % adapt quote style to current language
english=british,         % british english as default
threshold=1,             % set block quotations >1 line in display mode
maxlevel=4,              % max nesting level
]{csquotes}

\usepackage[
indentfirst=false,
vskip=0pt,               % by default would be \topsep + \partopsep.
]{quoting}

% tell csquotes to use quoting environment
% for \displayquote and \blockquote
\SetBlockEnvironment{quoting}

% if cite is issued by a csquote command
\renewcommand{\mkcitation}[1]{\space#1}

% I prefer double quotes as outer
\DeclareQuoteStyle{keithbritish}%  [variant]{style}
{\textquotedblleft}%                      opening outer mark
{\textquotedblright}%                     closing outer mark
[0.05em]%
{\textquoteleft}%                         opening inner mark
{\textquoteright}%                        closing inner mark

\ExecuteQuoteOptions{style=keithbritish}


\usepackage[
backend=biber,
%  style=ext-authoryear-comp,   % defined in biblatex-ext package
style=numeric
]{biblatex}

% more space between entries in bib
\setlength\bibitemsep{1.5\itemsep}

% kandrews: replace round brackets with square brackets in bibliography
% biblabeldate is a biblatex-ext feature
\DeclareFieldFormat{biblabeldate}{\mkbibbrackets{#1}}


% remove URL: from in front of URLs
\DeclareFieldFormat{url}{\url{#1}}
\DeclareFieldFormat{doi}{\doi{#1}}
\DeclareFieldFormat{isbn}{\isbn{#1}}
\DeclareFieldFormat{issn}{\issn{#1}}

% suppress urldate field
\AtEveryBibitem{\clearfield{urlyear}}

% remove In: from aricles and inproceedings entries
% https://tex.stackexchange.com/questions/10682/suppress-in-biblatex
\renewbibmacro{in:}{%
	\ifboolexpr{%
		test {\ifentrytype{article}}%
		or
		test {\ifentrytype{inproceedings}}%
	}{}{\printtext{\bibstring{in}\intitlepunct}}%
}

% make all entry titles italic
% (also removes quotation marks from around titles)
% https://tex.stackexchange.com/questions/311816/want-title-in-simple-numeric-not-italic-through-bibliography
\DeclareFieldFormat*{title}{\mkbibitalic{#1}}
\DeclareFieldFormat*{citetitle}{\mkbibitalic{#1}}

% make journal names non-italic
\DeclareFieldFormat{journaltitle}{#1\isdot}

% make proceedings names non-italic
\DeclareFieldFormat[inproceedings]{booktitle}{#1\isdot}

% use nth for edition
\DeclareFieldFormat{edition}{%
	\ifinteger{#1}
	{\nth{#1}~\bibstring{edition}}
	{#1\isdot}}

% overwrite some standard strings in english.lbx
\DefineBibliographyStrings{english}{%
	edition          = {Edition},
	mathesis         = {Master's Thesis},
	phdthesis        = {PhD\addabbrvspace Thesis},
}


% kandrews
% use Unix format for dates in biblio:
% 29 Dec 2015, 01 Oct 2018, etc.

% for now, define under lang english not british
% due to bug in biblatex 3.11

\DefineBibliographyStrings{english}{%
	january          = {Jan},
	february         = {Feb},
	march            = {Mar},
	april            = {Apr},
	may              = {May},
	june             = {Jun},
	july             = {Jul},
	august           = {Aug},
	september        = {Sep},
	october          = {Oct},
	november         = {Nov},
	december         = {Dec},
}

\DefineBibliographyExtras{english}{%
	% #1 = year, #2 = month, #3 = day
	\protected\def\mkbibdatelong#1#2#3{%
		\iffieldundef{#3}
		{}
		{\mkdayzeros{\thefield{#3}}%
			\iffieldundef{#2}{}{\nobreakspace}}%
		\iffieldundef{#2}
		{}
		{\mkbibmonth{\thefield{#2}}%
			\iffieldundef{#1}{}{\space}}%
		\iffieldbibstring{#1}{\bibstring{\thefield{#1}}}{\mkyearzeros{\thefield{#1}}}}%
	%
	\protected\def\mkbibdateshort#1#2#3{%
		\iffieldundef{#3}
		{}
		{\mkdayzeros{\thefield{#3}}%
			\iffieldundef{#2}{}{\nobreakspace}}%
		\iffieldundef{#2}
		{}
		{\mkbibmonth{\thefield{#2}}%
			\iffieldundef{#1}{}{\space}}%
		\iffieldbibstring{#1}{\bibstring{\thefield{#1}}}{\mkyearzeros{\thefield{#1}}}}%
}



\addbibresource{bibliography.bib}
\addbibresource{writing.bib}



% xurl provides better URL breaking than url
% load after biblatex
\usepackage[hyphens,obeyspaces]{xurl}
\def\UrlFont{\smaller\ttfamily}





% adapt pdftitle, pdfsubject, pdfauthor, pdfkeywords
% for your survey paper

\usepackage{ifpdf}

\ifpdf
% pdflatex
\usepackage[pdftex]{graphicx}
\DeclareGraphicsExtensions{.pdf,.jpg,.png}
\pdfcompresslevel=9
\pdfobjcompresslevel=1  % also compress PDF object streams except embedded PDFs
\pdfpageheight=297mm
\pdfpagewidth=210mm
\usepackage[            % hyperref should be last package loaded
unicode,
pdftex,
pdfversion=1.7,
pdftitle={Guidelines for Writing a Master's Thesis},
pdfsubject={Master's Thesis Template},
pdfauthor={Keith Andrews},
pdfkeywords={master's thesis, skeleton, guidelines, template},
bookmarks,
bookmarksnumbered,
linktocpage,
colorlinks,
linkcolor=darkred,
anchorcolor=red,
citecolor=darkgreen,
urlcolor=darkblue,
pdfstartview=Fit,              % initial view
pdfview=Fit,                   % view after following a link
pdfpagelayout=SinglePage,      % single page, no scrolling
pdfpagemode=UseOutlines,       % open bookmarks in Acrobat
plainpages=false,              % avoids duplicate page number problem
pdfpagelabels,                 % avoids duplicate page number problem
breaklinks=true,               % allow links exceeding a single line
]{hyperref}

\else
% latex
\usepackage[dvips]{graphicx}
\DeclareGraphicsExtensions{.eps}
\usepackage[dvips]{hyperref}
\fi

% export adjustbox keys to includegraphics
% must be after \usepackage{graphicx}
\usepackage[export]{adjustbox}    % valign=t, frame, ...

\usepackage{todonotes}

% macros and definitions

\newcommand\fname{\begingroup \smaller\urlstyle{tt}\Url}

\newcommand\vname{\begingroup \smaller\urlstyle{tt}\Url}


% for class names, define our own url style

\makeatletter  % protect @ names

% \url@letstyle: New URL sty to premit break at any letters.
% Based on \url@ttstyle

\def\Url@letdo{% style assignments for tt fonts or T1 encoding
\def\UrlBreaks{\do\a\do\b\do\c\do\d\do\e\do\f\do\g\do\h\do\i\do\j\do\k\do\l%
               \do\m\do\n\do\o\do\p\do\q\do\r\do\s\do\t\do\u\do\v\do\w\do\x%
               \do\y\do\z%
               \do\A\do\B\do\C\do\D\do\E\do\F\do\G\do\H\do\I\do\J\do\K\do\L%
               \do\M\do\N\do\O\do\P\do\Q\do\R\do\S\do\T\do\U\do\V\do\W\do\X%
               \do\Y\do\Z%
}%
\def\UrlBigBreaks{\do\.\do\@\do\\\do\/\do\!\do\_\do\|\do\%\do\;\do\>\do\]%
 \do\)\do\,\do\?\do\'\do\+\do\=\do\#\do\:\do@url@hyp}%
\def\UrlNoBreaks{\do\(\do\[\do\{\do\<}% (unnecessary)
\def\UrlSpecials{\do\ {\ }}%
\def\UrlOrds{\do\*\do\-\do\~}% any ordinary characters that aren't usually
\Urlmuskip = 0mu plus 1mu%
}

\def\url@letstyle{%
\@ifundefined{selectfont}{\def\UrlFont{\sf}}{\def\UrlFont{\sffamily}}\Url@letdo
}

\makeatother  % unprotect @ names


\newcommand\cname{\begingroup \smaller\urlstyle{let}\Url}


\newcommand{\imgcredit}[1]
{%
\small
[#1]
}


\newcommand{\chapquote}[2]
{%
\begin{quote}
\emph{%
``#1''%
}%
\begin{flushright}
{\scriptsize \sffamily [#2]}%
\end{flushright}
\end{quote}
}


% \urlfootnote{url}{day}{month}{year}
\newcommand{\murlfootnote}[4]{\footnote{\url{{#1}} (last visit {#4}-{#3}-{#2})}}
\newcommand{\murlfootnotebreak}[4]{\footnote{\url{{#1}}\\ \hspace*{6mm}(last visit {#4}-{#3}-{#2})}}

% change margin command
\def\changemargin#1#2{\list{}{\rightmargin#2\leftmargin#1}\item[]}
\let\endchangemargin=\endlist

% tikz stuff
\usepackage{tikz, wrapfig}
\usetikzlibrary{positioning, calc, shapes.geometric}


\begin{document}
	
	\unixdate
	
	\frontmatter
	
	\normalsize
	\pagestyle{empty}             % for title pages
	\pagenumbering{Roman}         % for pdf labels
	
	%----------------------------------------------------------------
%
%  File    			:  thesis-title.tex
%
%  Author  			: Benjamin Wunderling
%  Template from	: Keith Andrews, IICM, TU Graz, Austria	
% 
%----------------------------------------------------------------


% --- Main Title Page ------------------------------------------------

\iffalse
\vspace*{2cm}


\begin{center}
\begin{spacing}{1.1}
\Huge\sffamily\bfseries
Model Learning and Fuzzing of the IPsec-IKEv1 VPN Protocol\\
\end{spacing}

\vspace{3cm}

% \LARGE \sffamily Draft 0.9

\vspace{3cm}

{\LARGE\sffamily
Benjamin Wunderling
}
\end{center}



% --- English Title Page ------------------------------------------------


% based on:
% https://tu4u.tugraz.at/fileadmin/Studierende_und_Bedienstete/Formulare/Diplomarbeit_Vorlage.pdf



\cleardoublepage
\fi

\vspace*{-3cm}

\begin{center}
\includegraphics[height=1cm]{diagrams/tugraz-logo.pdf}

\vspace{3cm}

{\Large\sffamily Benjamin Wunderling, BSc}

\vspace{1cm}

\begin{spacing}{1.1}
\huge\sffamily\bfseries
Model Learning and Fuzzing of the IPsec-IKEv1 VPN Protocol\\
\end{spacing}

\vspace{3cm}


{\Large\sffamily\bfseries Master's Thesis}

\vspace{5mm}

{\small\sffamily to achieve the university degree of}

\vspace{5mm}

{\normalsize\sffamily Diplom-Ingenieur}

\vspace{5mm}

{\normalsize\sffamily
Master's Degree Programme: Computer Science
}


\vspace{1cm}

{\small\sffamily submitted to}

\vspace{5mm}

{\large\sffamily\bfseries Graz University of Technology}



\vspace{1cm}

{\small\sffamily Supervisor}

\vspace{5mm}

{\normalsize\sffamily
Ao.Univ.-Prof.\ Dipl-Ing.\ Dr.\ Bernhard K.~Aichernig \\
Institute of Software Technology (IST)
}


\vspace{1cm}

{\normalsize\sffamily Graz, \thisdate}



\vfill


\end{center}





% --- Pledge ----------------------------------------------------

\cleardoublepage

\vspace*{2cm}


% adapted from:
% https://tu4u.tugraz.at/fileadmin/Studierende_und_Bedienstete/Formulare/Diplomarbeit_Vorlage.pdf
% https://tu4u.tugraz.at/fileadmin/Studierende_und_Bedienstete/Forms/Diploma_thesis_template.pdf
% Email vom 2. Sept. 2015

% and

% Beschluss der Curricula-Kommission für Bachelor-,
% Master- und Diplomstudien vom 10.11.2008
% Genehmigung des Senates am 1.12.2008



\subsection*{Statutory Declaration}
\noindent
\textit{
I declare that I have authored this thesis independently, that I have
not used other than the declared sources / resources, and that I have
explicitly indicated all material which has been quoted either
literally or by content from the sources used. The document uploaded
to TUGRAZonline is identical to the present thesis.}

\vspace{1cm}


\begin{otherlanguage}{austrian}

\subsection*{Eidesstattliche Erklärung}
\noindent
\textit{
Ich erkläre an Eides statt, dass ich die vorliegende Arbeit
selbstständig verfasst, andere als die angegebenen Quellen/Hilfsmittel
nicht benutzt, und die den benutzten Quellen wörtlich und inhaltlich
entnommenen Stellen als solche kenntlich gemacht habe. Das in
TUGRAZonline hochgeladene Dokument ist mit der vorliegenden
Arbeit identisch.}

\vspace{2cm}

\noindent
\parbox[top]{4cm}{
\begin{center}
\underline{\hspace*{4cm}} \\
Date/Datum
\end{center}
}
%
\hfill
%
\parbox[top]{6cm}{
\begin{center}
\underline{\hspace*{6cm}} \\
Signature/Unterschrift
\end{center}
} 

\end{otherlanguage}




% --- English Abstract ----------------------------------------------------


\cleardoublepage

\vspace*{2cm}

\begin{center}
{\Large\sffamily\bfseries Abstract}
\end{center}
% Motivation
In light of the growing number of individuals working from home, the importance of secure \ac{vpn} communication protocols has increased significantly. Consequently, the need for rigorous security testing of \ac{vpn} software has become more critical than ever before. 
In this thesis, we employ \ac{aal} to demonstrate how behavioral models of two black-box \ac{ipsec} \ac{vpn} implementations can be automatically learned. We subsequently utilize the learned models to facilitate security testing the \ac{ipsec} implementations through model-based fuzzing. 
More specifically, we first learn behavioral models of strongSwan and libreswan \ac{ipsec} servers using both the $L^*$ and $KV$ model learning algorithms. Learned models of both \ac{ipsec} implementations are presented and the utilized learning algorithms are compared and evaluated. The learned models are then used to perform model-based fuzzing of the respective implementations. New states are detected by comparing against the previously-learned models. Additionally, we employ search-based methods, including a genetic algorithm-based approach, aided by the learned models, in order to enhance the effectiveness of fuzzing. Our fuzzing analysis uncovered several new security issues, including several standard violations and a potential deadlock state. Furthermore, our model learning approach led to the discovery of a bug in a cryptographic Python library.

          % Title Pages, Abstracts, Pledge
	
	
	\cleardoublepage
	\pagestyle{plain}             % for preliminary pages
	\pagenumbering{roman}         % for preliminary pages
	
	
	% TODO ka is this necessary?  see iaweb.tex
	% \phantomsection    % so hyperref pdf link jumps to correct page??
	
	\begin{spacing}{0.8}
		\tableofcontents
	\end{spacing}
	\addcontentsline{toc}{chapter}{Contents}
	
	\cleardoublepage
	\begin{spacing}{0.8}
		\listoffigures
	\end{spacing}
	\addcontentsline{toc}{chapter}{List of Figures}
	
	\cleardoublepage
	\begin{spacing}{0.8}
		\listoftables
	\end{spacing}
	\addcontentsline{toc}{chapter}{List of Tables}
	
	\cleardoublepage
	\begin{spacing}{0.8}
		\lstlistoflistings
	\end{spacing}
	\addcontentsline{toc}{chapter}{List of Listings}
	
	
	
	\cleardoublepage
	%----------------------------------------------------------------
%
%  File    :  thesis-acknowl.tex
%
%  Author  :  Benjamin Wunderling, TU Graz, Austria
%
%----------------------------------------------------------------


\chapter*{Acknowledgements}
\addcontentsline{toc}{chapter}{Acknowledgements}

First and foremost, I would like to thank my coadvisor at TU Graz, Dipl.-Ing. Andrea Pferscher, whose patience, advice and draft corrections made this thesis possible. 

I would also like to thank Ao.Univ.-Prof. Dipl-Ing. Dr. Bernhard K.~Aichernig for his support of my choice in topic and his invaluable feedback.

Last but not least, a special thanks goes to the most important person in my life, Alex, my better half, whose constant motivation and food helped keep me sane during long days of debugging.


\vspace{2cm}


\begin{flushright}
Benjamin Wunderling \\[1ex]
{\small Graz, Austria, \thisdate}
\end{flushright}

        	% Acknowledgements
	
	\cleardoublepage
	%----------------------------------------------------------------
%
%  File    :  vpn_creditss.tex
%
%  Author  :  Benjamin Wunderling, TU Graz, Austria
% 
%----------------------------------------------------------------


\chapter*{Credits}
\addcontentsline{toc}{chapter}{Credits}

I would like to thank the following individuals and organisations for
permission to use their material:
\begin{itemize}
\item The thesis was written using Keith Andrews' skeleton
  thesis \parencite{KeithThesis}.

\end{itemize}


        	% Credits
	
	
	
	\mainmatter
	
	\cleardoublepage
	\pagestyle{main}            % for main pages
	\pagenumbering{arabic}      % for main pages
	
	% Part 1 Embedding
	
	\cleardoublepage
	%----------------------------------------------------------------
%
%  File    :  thesis-embed.tex
%
%  Author  :  Keith Andrews, IICM, TU Graz, Austria
% 
%  Created :  22 Feb 96
% 
%  Changed :  19 Feb 2004
% 
%----------------------------------------------------------------

\chapter{Introduction}

\label{chap:Introduction}

%TODO: expand on
% Background
\section{Motivation}
\ac{vpn} are used to allow secure communication over an insecure channel. They function by creating a secure encrypted tunnel through which users can send their data. Example use cases include additional privacy from prying eyes such as Internet Server Providers, access to region-locked online content and secure remote access to company networks. The importance of \ac{vpn} software has increased dramatically since the beginning of the COVID-19 pandemic due to the influx of people working from home \cite{DBLP:journals/cacm/FeldmannGLPPDWW21}. This makes finding vulnerabilities in \ac{vpn} software more critical than ever. % state reports on damages by vpn errors? 
% IPSEC
\ac{ipsec} is a popular \ac{vpn} protocol suite and most commonly uses the \ac{ike} protocol to share authenticated keying material between involved parties. Therefore, \ac{ike} and \ac{ipsec} are sometimes used interchangeably. We will stick to the official nomenclature of using \ac{ipsec} for the full protocol and \ac{ike} for the key exchange only. \ac{ike} has two versions, \ac{ike}v1~\cite{rfc:ikev1} and \ac{ike}v2~\cite{rfc:ikev2}, with \ac{ike}v2 being the newer and recommended version, according to a report by the National Institute of Standards and Technology~\cite{nist791491}. However, despite \ac{ike}v2 supposedly replacing its predecessor, \ac{ike}v1, sometimes also called Cisco \ac{ipsec}, is still in widespread use today. This is reflected by the company AVM to this day only offering \ac{ike}v1 support for their popular FRITZ!Box routers \cite{avm2022}. Additionally, \ac{ike}v1 is also used for the L2TP/\ac{ipsec} protocol, one of the most popular \ac{vpn} protocols according to NordVPN \cite{nordvpn2021}. The widespread usage of \ac{ipsec}-\ac{ike}v1, combined with its relative age and many options makes it an interesting target for security testing.


\section{Research Problems and Goals}
% Automata Learning
State machines of protocol implementations are useful tools in state-of-the-art software testing. They have, e.g., been used to detect specific software implementations, or to generate test cases automatically \cite{pferscher2021fingerprinting, pferscher2022fuzzing}. Mealy machines are a type of state machine that can be used to describe the output of a system when given the current state and an external input. Often we are interested in testing software without knowing its exact inner workings, e.g. due to the software being closed-source. We call these systems black-box systems. It is not unusual for VPN software to be closed-source and therefore a black-box system for testers. However, despite lacking information on the inner structure of a black-box system, the state machine of the system can still be extracted. One method of generating a state machine of such a system is to use active automata learning. A notable example of an active automata learning algorithm is the $L^*$ algorithm by Angluin \cite{angluin1987learning}. In $L^*$, a learner queries the \ac{sul} and constructs a behavioral model of the \ac{sul} through its responses. 

% Problem / motivation
By combining automata learning with fuzzing or similar software testing techniques, network protocols can be extensively and automatically tested without requiring access to their source code. Guo et al.~\cite{guo2019model} tested \ac{ipsec}-\ac{ike}v2 using automata learning and model checking, however so far, no studies have focused on \ac{ike}v1 in the context of automata learning. Therefore our goal was to black-box test the \ac{ipsec}-\ac{ike}v1 protocol using automata learning in combination with automata-based fuzzing. We used the active automata learning framework \textsc{AALpy} \cite{muvskardin2022aalpy} with a custom mapper to learn the state machines of various \ac{ipsec}-\ac{ike}v1 server implementations. We then further utilized the learned models for fuzzing and fingerprinting. % TODO: add any findings here

% Expand this section more
\section{Structure}
This thesis is structured as follows. Chapter~\ref{chap:Related} gives an overview of the related literature. Chapter~\ref{chap:Preliminaries} introduces necessary background knowledge, covering the \ac{ipsec}-\ac{ike}v1 protocol, Mealy machines, automata learning and fuzzing. Our learning setup, custom mapper and fuzzing methodology are presented in Chapter~\ref{chap:Learning}. In Chapter~\ref{chap:Evaluation}, learned models and the results of the fuzzing tests are showcased and analyzed.
Finally, Chapter~\ref{chap:Conclusion} summarizes the thesis and discusses future work.          	% Introduction: 1-2p (split into motivation, Research Problem and Goals and Structure)
	
	\cleardoublepage
	%----------------------------------------------------------------
%
%  File    :  vpn_related.tex
%
%  Author  :  Keith Andrews, IICM, TU Graz, Austria
% 
%  Created :  22 Feb 96
% 
%  Changed :  19 Feb 2004
% 
%----------------------------------------------------------------

\chapter{Related}

\label{chap:Related}
% TODO: check if Feldman is correct
% TODO: add a summary of mentioned related work with each citation, total length should be about 2 pages
The aim of this chapter is to give a brief overview of related work, focusing mainly on automata learning and testing of secure communication protocols.
1987 
The concept of learning through the means of membership and equivalence queries was introduced in 1987 by Angluin \parencite{angluin1987learning}. Angluin presented an algorithm for learning regular languages from queries and counterexamples, called $L^*$. In it, a student questions a teacher and constructs a hypothesis based on its responses. The hypothesis is then tested through equivalence queries which check if the hypothesis correctly matches the regular language being learned. While the L* algorithm was originally designed to learn deterministic finite automata (dfa), it can be simply extended to work for Mealy machines by making use of the similarities between dfa and Mealy machines, as shown by Steffen et al. \parencite{steffen2011}. Over time, many related and improved algorithms were published, such as the one proposed by Rivest and Schapire in 1993 in which homing sequences were used to infer finite automata \parencite{Rivest1993Inference}. Another, more recent algorithm came in the form of a redundancy-free active automata learning approach titled TTT by Isberner et al \parencite{Isberner2014TTT}. In this algorithm, essential data is stored in three tree data structures, stripping away unessential information. %TODO: mention version of algorithm used by AALpy

Model learning network protocols for the purpose of testing is a more recent development, with models of protocols like SSH \parencite{fiteruau2017model}, or TCP \parencite{fiteruau2016combining} being learned and used for model checking. Both Novickis et al.~\parencite{novickis2016protocol} and Daniel et al.~\parencite{daniel2018inferring} learned models of the related OpenVPN protocol and used the learned models to perform protocol fuzzing. In a work by Pferscher and Aichernig \parencite{pferscher2021fingerprinting}, learned models were used to fingerprint Bluetooth Low Energy devices (BLE), showing yet another possible use case of automate learning. Guo et al. \parencite{guo2019model} used automata learning to learn and test the IPsec-IKEv2 protocol. They used the LearnLib~\footnote{https://learnlib.de/} library for automata learning and performed model checking of the protocol, using the learned state machine. In contrast, our work focuses on the IPsec-IKEv1 protocol, the predecessor of IPsec-IKEv2, which, to the best of our knowledge, has not yet been tested with methods utilizing automata learning. The protocols differ greatly on a packet level, with IKEv1 needing more than twice the amount of packets to establish a connection than IKEv2.
Additionally we used the \textsc{AALpy}~\footnote{https://github.com/DES-Lab/AALpy} library for automata learning and focused on fuzzing and fingerprinting as opposed to model checking.
			% Related work: 1 page?
	
	\cleardoublepage
	%----------------------------------------------------------------
%
%  File    :  vpn_prelim.tex
%
%  Author  :  Keith Andrews, IICM, TU Graz, Austria
% 
%  Created :  22 Feb 96
% 
%  Changed :  19 Feb 2004
% 
%----------------------------------------------------------------

\chapter{Preliminaries}

\label{chap:Preliminaries}

\section{Mealy Machines}
% TODO: Expand, add grafic or listing, see other papers for inspiration
Mealy machines are finite state machines where each output transition is defined by the current state and an input. More formally, a Mealy machine is defined as a 6-tuple $M = \{S, S_0, \Sigma, \Lambda, T, G\}$, where $S$ is a finite set of states, $S_0 \in S$ is the initial state, $\Sigma$ is a finite set called the input alphabet, $\Lambda$ is a finite set called the output alphabet, $T$ is the transition function $G: S \times \Sigma \rightarrow \Lambda$ which maps a state and an element of the input alphabet to another state in $S$ and $G$ is the output function $T: S \times \Sigma \rightarrow S$ which maps a state-input alphabet pair to an element of the output alphabet $\Lambda$. We use Mealy machines to model the state of learned IPsec implementations.

\section{Automata Learning}

% TODO: More on automata learning, in particular cover L* in detail. Consider what is repeated here from Related chapter.
Automata learning refers to methods of learning the state model, or automaton, of a system through an algorithm or process. We differentiate between active and passive automata learning. In passive automata learning (PAL), models are learned based on a given data set describing the behavior of the SUL, e.g. log files. In contrast, in active automata learning (AAL) the SUL is queried directly. In this paper, we will focus on AAL and will, moving on, be referring to it as automata learning or AAL interchangeably. 

One of the most influential AAL algorithms was introduced in 1987 through a paper by Dana Angluin, titled ``Learning regular sets from queries and counterexamples''~\parencite{ANGLUIN198787}. In this seminal paper, Angluin introduced the $L^*$ algorithm, variants of which are still used for learning deterministic automata to this day, for example by the AAL python library \textsc{AALpy} \parencite{muvskardin2022aalpy}. While the original $L^*$ algorithm was designed to learn deterministic finite automata (DFA), the algorithm can be extended to learn Mealy machines \parencite{Niese2003AnIA}. While many modern implementations, including \textsc{AALpy} use improved versions of $L^*$, fundamentally they still resemble the original algorithm by Angluin. The base $L^*$ algorithm is briefly explained below. TODO: does AALpy version use homing sequences? --> Rivest1993Inference

Another popular algorithm in the field of AAL is the $KV$ algorithm by Kearn's and Vazirani \parencite{KV1994}. Published later than $L^*$, it boasts a more compact method of representing learned data called a classification tree. This, on average, leads to the $KV$ algorithm requiring less membership queries than $L^*$ to learn a system. Especially for learning internet protocols and other systems where communication with the SUL can be very time consuming, this can result in a significant performance increase.

% TODO: expand greatly, go into more detail here
\subsection{L\textsuperscript{*}}
$L^*$ uses a Minimally Adequate Teacher (MAT) model in which a learner queries a teacher in order to learn an unknown regular language $L$. Queries are built using a fixed input alphabet $\Sigma$ where $L \subseteq \Sigma^*$ must hold. The teacher must respond to two types of queries posed by the learner, namely membership and equivalence queries. Membership queries consist of a word $s \in \Sigma^*$ and must be answered with either ``yes'' if $s \in L$, or ``no'' if not. In other words, membership queries are used to check if a given word is part of the language being learned. Equivalence queries on the other hand, consist of a regular language $L_{prop}$, proposed by the learner. The teacher must answer with  ``yes'' if $L_{prop} \equiv L$, or returns a counterexample $c$ proving the two languages are different, so $c \in L(S) \iff c \notin L$. In other words, equivalence queries are used to verify if the learner has successfully learned the target language $L$ or if not, return a counterexample detailing the differences. The results of the membership queries are stored in an observation table $O = (S,E,T)$, where $S$ is a prefix-closed set of strings representing candidates for states of $L_{prop}$, $E$ a suffix-closed set of strings used to distinguish between candidates and $T$ a transition function $(S \cup S \cdot \Sigma) \cdot E \rightarrow {0,1}$. Essentially, if visualized as a 2D array where the rows are labeled with elements in $(S \cup S \cdot \Sigma)$ and columns with elements in $E$, the entries in the table are ones, if the word created by appending the row-label to the column-label is accepted by $L$ and zeros if not. 
The goal of $L^*$ is to learn a DFA acceptor for $L$ using the observation table. S-labeled rows correspond to states in the acceptor under construction. E-labeled columns represent individual membership query results. For the observation table to be transformable into a DFA acceptor, it must first be closed and consistent.


\begin{lstlisting}[mathescape=true, float=ht, caption=$L^*$ algorithm, label=lst:lstar]
	$\mathbf{Initialization}$: 
	Set observation table $O = (S,E,T)$ with $S,E = \{\epsilon\}$.
	$populate(O)$.
	
	$\mathbf{repeat}$:
		$\mathbf{while}$ $O$ is not closed or not consistent $\mathbf{do}$
			$\mathbf{if}$ $O$ is not closed $\mathbf{then}$
				choose $s_1 \in S, \sigma \in \Sigma$ such that
				$row(s_1 \cdot \sigma) \neq row(s) \; \forall s \in S$
				add $s_1 \cdot \sigma$ to $S$
				$populate(O)$
			$\mathbf{end}$
			$\mathbf{if}$ $O$ is not consistent $\mathbf{then}$
				choose $s_1, s_2 \in S, \sigma \in \Sigma$ and $e \in E$ such that
				$row(s_1) = row(s_2)$ and $T(s_1 \cdot \sigma \cdot e) \neq T(s_2 \cdot \sigma \cdot e)$
				add $\sigma \cdot e$ to $E$
				$populate(O)$
			$\mathbf{end}$		
		$\mathbf{end}$
		Construct $L_{prop}$ from $O$ and perform an equivalence query.	
		$\mathbf{if}$ query returns a counterexample $c$ $\mathbf{then}$
			add all prefixes of $c$ to $S$
			$populate(O)$
		$\mathbf{end}$
	$\mathbf{until}$ teacher replies "yes" to equivalence query $L_{prop} \equiv L$
	$\mathbf{return}$ $L_{prop}$
		
\end{lstlisting}

Closed is defined as for all $t \in S \cdot \Sigma$ there exists an $s \in S$ so that $row(t) = row(s)$. In other words, that no new information is gained by expanding the $S$-set by any word in $\Sigma$. If an observation is not closed, it is fixed by adding $t$ to $S$ and updating the table rows through more membership queries. 
Consistent means, that $\forall s_1, s_2 \,|\, row(s_1) = row(s_2) \implies \forall \sigma \in \Sigma \,|\, row(s_1 \cdot \sigma) = row(s_2 \cdot \sigma)$, or in other words, appending the same word to identical states should not result in different outcomes. If an observation table is inconsistent, it is made consistent again by adding another column to the table with the offending $\sigma$ as its label and again updating the table rows through more membership queries. 

Listing~\ref{lst:lstar} shows the workings of the basic $L^*$ algorithm by Angluin. The function $populate(O)$ extends $T$ to $(S \cup S \cdot \Sigma) \cdot E$ by asking membership queries for all table entries still missing membership information. At the start of the algorithm, the observation table is initialized with $S = E = \{\epsilon\}$. Next, until a equivalence query succeeds, the observation table is repeatedly brought to a closed and consistent state by expanding the $S$ and $E$ sets respectively. Once both closed and consistent, $L_{prop}$ is constructed from $O$ and used in an equivalence query. If the equivalence query returns ``yes", the algorithm terminates, returning the learned DFA. If not, the returned counterexample is used to update the observation table and the algorithm loops back to line 5. 


\subsection{KV}
Another notable AAL algorithm is the KV algorithm published in 1994 by Kearns and Vazirani \cite{KV1994}. It is designed to work in the same learner-teacher framework as $L^*$, but was designed to minimize the amount of membership queries needed to learn a finite automaton $M$. The KV algorithm does this by organizing learned information in an ordered binary tree called a classification tree $C_T$ as opposed to the table structure utilized by $L^*$. Intuitively, $L^*$ must perform membership queries for every entry in the observation table to differentiate between possible states, whereas $KV$ requires only a subset to distinguish them. 

In the $KV$ algorithm, learned data is stored in two sets called the access strings set $S$ and the distinguishing strings set $D$. Every string $s \in S$ represents a distinct and unique state of the automaton $M$. In other words, any $s$ when applied starting in the initial state of $M$ leads to a unique state $M[s]$. The distinguishing strings set is defined as the set of strings $d \in D$ where for each pair $s,s' \in S, s \neq s'$ there exists a $d \in D$ such that either $M[s \cdot d]$ or $M[s' \cdot d]$ is an accepting state. $D$ is used to ensure that their are no ambiguous states. The sets $S,D$ are organized in a binary tree called the classification tree $C_T$ where parent nodes are strings from $D$ and the leaf nodes are strings from $S$. The root node is set to the empty string $\lambda$. For each node of the tree, starting from the root node, each right subtree contains access strings to accepting states while left subtrees contain access strings to rejecting states of $M$.
Given a new string $s'$, we simply start at the root nodes, then sift down the tree by executing a membership query for $s' \cdot \lambda_1$ and depending on if the query returns ``yes" or ``no" continuing with the left or right subtree until we reach a leaf node labeled with $s$. If $s' = s$ then the states are equivalent, otherwise the classification tree is updated to include another leaf node representing the newly learned distinct state $s'$. The main learning loop of the $KV$ algorithm is shown in more detail in Listing~\ref{lst:kv}. Following the initialization of the classification table, new states learned from counterexamples are repeatedly added until an equivalence query is successful. The $Update(C_T,c)$ function adds a new leaf to the $C_T$ based on a counterexample $c$ returned from an equivalence query.

\begin{lstlisting}[mathescape=true, float=ht, caption=$KV$ algorithm, label=lst:kv]
	$\mathbf{Initialization}$: 
		Set root node of $C_T$ to $\epsilon$. 
		Perform membership query on $\epsilon$ to determine if the initial state is accepting or not.
		Construct hypothesis automaton $\hat M$ consisting of only the initial state, with self-transitions for all other transitions.
		Add two access strings $\epsilon$ and the counterexample string $c$.
	
	$\mathbf{repeat}$:
		Construct hypothesis automaton $\hat M$ from $C_T$.
		Equivalence query($\hat M$)
		$\mathbf{if}$: query returns "yes" $\mathbf{then}$
			$\mathbf{return}$ $\hat M$
		$\mathbf{end}$
		
		$Update(C_T,c)$
	$\mathbf{end}$
	
\end{lstlisting}


\section{Fuzzing}
% TODO: leave till we actually start this / could write the basics now --> show how model-based fuzzing can worl

\section{IPsec}
% TODO: More on VPNs in general

Virtual Private Networks (VPN) are used to extend and or connect private networks across an insecure channel (usually the public internet). They can be used e.g. to gain additional privacy from prying eyes such as Internet Server Providers, access to region-locked online content or secure remote access to company networks. Many different VPN protocols exit, including PPTP, OpenVPN and Wireguard. IPsec or IP Security, is a VPN layer 3 protocol used to securely communicate over an insecure channel. It is based on three sub-protocols, the Internet Key Exchange (IKE) protocol, the Authentication Header (AH) protocol and the Encapsulating Security Payload (ESP) protocol. IKE is mainly used to handle authentication and to securely exchange as well as manage keys. Following a successful IKE round, either AH or ESP is used to send packets securely between parties. The main difference between AH and ESP is that AH only ensures the integrity and authenticity of messages while ESP also ensures their confidentiality through encryption.

\begin{figure}[H]
\begin{centering}
	\begin{tikzpicture}[scale=1]
		\draw (-3,0) -- (-3,-9.2) (3,0) -- (3,-9.2);
		\node at (-3,.3) {Initiator};
		\node at (3,.3) {Responder};
		\draw[->] (-3,-1) -- node[midway,above] {ISAKMP SA \{proposals\}} (3,-1);
		\draw[<-] (-3,-2) -- node[midway,above] {ISAKMP SA \{proposal\}} (3,-2);
		\draw[->] (-3,-3) -- node[midway,above] {KE $\{pkey_i, nonce_i\}$} (3,-3);
		\draw[<-] (-3,-4) -- node[midway,above] {KE $\{pkey_r, nonce_r\}$} (3,-4);
		\draw[->] (-3,-5) -- node[midway,above] {AUTH $\{hash_i\}$} (3,-5);
		\draw[<-] (-3,-6) -- node[midway,above] {AUTH $\{hash_r\}$} (3,-6);
		\draw[->] (-3,-7) -- node[midway,above] {IPSEC SA \{proposals\}} (3,-7);
		\draw[<-] (-3,-8) -- node[midway,above] {IPSEC SA \{proposal\}} (3,-8);
		\draw[->] (-3,-9) -- node[midway,above] {ACK} (3,-9);
	\end{tikzpicture}
	\caption{IKEv1 between two parties}
	\label{fig:IKEv1}
\end{centering}
\end{figure}

The IKEv1 protocol works in two main phases, both relying on the Internet Security Association and Key Management Protocol (ISAKMP). Additionally, phase one can be configured to proceed in either Main Mode or Aggressive Mode. A typical exchange between two parties, an initiator and a responder, using Main Mode for phase one, can be seen in Figure \ref{fig:IKEv1}. In phase one (Main Mode), the initiator begins by sending a Security Association (SA) to the responder. A SA essentially details important security attributes required for a connection such as the encryption algorithm and key-size to use, as well as the authentication method and the used hashing algorithm. These options are bundled in containers called proposals, with each proposal describing a possible security configuration. While the initiator can send multiple proposals to give the responder more options to choose from, the responder must answer with only one proposal, provided both parties can agree upon one of the suggested proposals. This initial communication is denoted as \emph{ISAKMP SA} in Figure~\ref{fig:IKEv1}. Subsequently, the two parties perform a Diffie-Hellman key exchange, denoted as \emph{KE}, and send each other nonces used to generate a shared secret key \emph{SKEYID} as detailed in Listing~\ref{lst:keying}. PSK refers to the pre-shared key, Ni/Nr to the initiator/responder nonce and CKY-I/CKY-R to the initiator/responder identifier cookie. Note that IKEv1 allows using various different authentication modes aside from PSK, including public key encryption and digital signatures. \emph{SKEYID} is used as a seed key for all further session keys \emph{SKEYID\_d}, \emph{SKEYID\_a}, \emph{SKEYID\_e}, with $g^{xy}$ referring to the previously calculated shared Diffie-Hellman secret and prf to a pseudo-random function (in our case, HMAC). Following a successful key exchange, all further messages of phase one and two are encrypted using a key derived from \emph{SKEYID\_e} and \emph{SKEYID\_a} for authentication. Finally, in the last section of phase one \emph{AUTH}, both parties exchange and verify hashes to confirm the key generation was successful. Once verification succeeds, a secure channel is created and used for phase two communication. If phase one uses Aggressive Mode, then only three packets are needed to reach phase two. While quicker, the downside of Aggressive Mode is that the communication of the hashed authentication material happens without encryption. This means, that using short pre-shared keys in combination with Aggressive Mode is inherently insecure, as the unencrypted hashes are vulnerable to brute-force attacks provided a short key-size~\footnote{https://nvd.nist.gov/vuln/detail/CVE-2018-5389}. The shorter phase two (Quick Mode) begins with another SA exchange, labeled with \emph{IPSEC SA} in Figure~\ref{fig:IKEv1}. This time, however, the SA describes the security parameters of the ensuing ESP/AH communication and the data is sent authenticated and encrypted using the cryptographic material calculated in phase one. This is followed by a single acknowledge message, \emph{ACK}, from the initiator to confirm the agreed upon proposal. After the acknowledgment, all further communication is done via ESP/AH packets, using \emph{SKEYID\_d} as keying material.

\begin{lstlisting}[float=ht, caption=IKE Keying, label=lst:keying]
	# For pre-shared keys: 
	SKEYID = prf(PSK, Ni_b | Nr_b)
	
	# to encrypt non-ISAKMP messages (ESP)
	SKEYID_d = prf(SKEYID, g^xy | CKY-I | CKY-R | 0)
	
	# to authenticate ISAKMP messages
	SKEYID_a = prf(SKEYID, SKEYID_d | g^xy | CKY-I | CKY-R | 1)
	
	# for further encryption of ISAKMP messages in phase two
	SKEYID_e = prf(SKEYID, SKEYID_a | g^xy | CKY-I | CKY-R | 2)
\end{lstlisting}

In addition to the packets shown in Figure~\ref{fig:IKEv1}, IKEv1 also specifies and uses so called ISAKMP Informational Exchanges. Informational exchanges in IKEv1 are used to send ISAKMP Notify or ISAKMP Delete payloads. Following the key exchange in phase one, all Informational Exchanges are sent encrypted and authenticated. Prior, they are sent in plain. ISAKMP Notify payloads are used to transmit various error and success codes, as well as for keep-alive messages. ISAKMP Delete is used to inform the other communication partner, that a SA has been deleted locally and request that they do the same, effectively closing a connection. 

Compared to other protocols, IPsec offers a high degree of customizability, allowing it to be fitted for many use cases. However, in a cryptographic evaluation of the protocol, Ferguson and Schneier \textcite{ferguson1999cryptographic} criticize the complexity arising from the high degree of customizability as the biggest weakness of IPsec. To address its main criticism, IPsec-IKEv2 was introduced in RFC 7296 to replace IKEv1 \parencite{kaufman2014internet}. Nevertheless, IPsec-IKEv1 is still in wide-spread use to this day, with the largest router producer in Germany, AVM, still only supporting IKEv1 in their routers \parencite{avm2022}. We use IPsec-IKEv1 with Main Mode and ESP in this paper and focus on the IKE protocol as it is the most interesting from an AAL and security standpoint. % this part could maybe be moved to the introduction, not sure
			% Preliminary work: 7-10 pages
	
	% Part 2 Original Work
	\cleardoublepage
	%----------------------------------------------------------------
%
%  File    :  vpn_setup.tex
%
%  Author  :  Keith Andrews, IICM, TU Graz, Austria
% 
%  Created :  22 Feb 96
% 
%  Changed :  19 Feb 2004
% 
%----------------------------------------------------------------

\chapter{Environment Setup} \label{chap:Setup}

This chapter provides an in-depth discussion of the environment used for both learning and fuzzing, focusing on its setup and configuration. The chapter begins with a detailed examination of the \ac{vm} setup, providing enough information to allow for the creation of a functionally identical \ac{vm} environment. Relevant networking optimizations and design choices are highlighted. Following the discussion of the \ac{vm} configuration, the installation and configuration of the two utilized \ac{ipsec} \ac{vpn} servers, strongSwan and libreswan, are examined in detail. Special focus is given to providing a comprehensive overview of the relevant \ac{ipsec} configuration file options, including which options map to which keywords for the two servers respectively. Additionally, the most notable differences between the two servers are showcased and discussed.

\section{VM Setup} \label{sec:vm_setup}
All model learning and testing took place in a virtual environment using two VirtualBox 6.1 \acp{vm} running standard Ubuntu 22.04 LTS distributions (x64). Each \ac{vm} was allotted \SI{4}{\giga\byte} of memory and one CPU core. To set up the base Ubuntu \ac{vm} using VirtualBox, we downloaded the Ubuntu image from the official source~\footnote{\url{https://ubuntu.com/download/desktop}} and created a new generic Ubuntu (x64) \ac{vm} in VirtualBox, specifying the downloaded Ubuntu image as the target ISO image. Next, we configured the hardware settings as detailed above and set the network to use NAT mode to be able to use the host computers internet connection to install updates and the \ac{vpn} software. Furthermore, power-saving options and similar potential causes of disruptions were disabled within the \acp{vm} as well as on the host computer during testing.
Additionally, a shared folder was created on each \ac{vm}, linked to the local development folder on the host computer. This allowed for very easy testing, as there was no need to copy over Python code after every change. Instead it could simply be run from the mounted folder directly. Design-wise, for each pair of \acp{vm}, one was designated as the \ac{vpn} initiator and one as the responder, to create a typical client-server setup. Following the installation and configuration of the \ac{vpn} software (explained in detail in Section~\ref{sec:vpn_setup}), all that remained was the final network configuration. 

VirtualBox supports many networking modes for its \acp{vm}, including several that allow for inter-\ac{vm} communication. These include host-only, internal, bridged and NAT-network networking modes.
As we wished to minimize external traffic, we configured the \acp{vm} to use the internal networking mode, as it is the only one that solely supports \ac{vm}-\ac{vm} communication.
The internal networking mode works by creating named internal networks that one can assign the network adapters of individual \acp{vm} to. All \acp{vm} within the same internal network can communicate freely with one-another, but not with the host computer, or any other network for that matter. We created a separate internal test network for each client-server pair of \acp{vm}, ensuring that all communication is isolated to the two involved parties. One important VirtualBox setting, is to change the network adapter from the default Intel, to the paravirtualized network adapter. Paravirtualized means, that instead of virtualizing networking hardware, VirtualBox simply ensures that packets arrive at their designated destination, through a special software interface in the guest operating system. This leads to a noticeable network performance increase. Within the guest Ubuntu installations, we configured the server to use the 10.0.2.1 and the client to use the 10.0.2.2 IP addresses respectively. We use the 2.0.2.0 network (255.255.255.0 subnet mask) with 2.0.2.0 as our default gateway. VirtualBox handles all of the internal routing, provided the two \acp{vm} are in the same internal network. Libreswan requires an additional second internal network with a different IP range to be configured. It is required for SSH-based resetting of the server from the client. This is discussed in Chapter~\ref{chap:Learning} in more detail.

As we use a separate internal network for each pair of \acp{vm}, we can leave the IP configurations identical between pairs and only have to change the used internal network name. This makes cloning pairs of \acp{vm} very practical, allowing for numerous identical test setups to be created and run at the same time, limited solely by the computing power of the host machine. Figure~\ref{fig:AALSetup} shows such a pair of \acp{vm} being run side by side.

\begin{figure}
	\centering
	\includegraphics[width=\linewidth]{images/VM_setup}
	\caption{Pair of VMs running side by side, showing log output. \\Responder/server on the left, initiator/client on the right.}
	\label{fig:vmsetup}
\end{figure}


\section{VPN Configuration} \label{sec:vpn_setup}
The two \ac{ipsec} implementations learned and tested were strongSwan and Libreswan. Both are popular open source \ac{ipsec} implementations, with strongSwan featuring support for Linux, Android, FreeBSD, Apple OSX and Windows~\cite{doc:strongswan} and being the more widespread choice of the two. The libreswan \ac{ipsec} implementation supports Linux, FreeBSD and Apple OSX~\cite{doc:libreswan}. Both projects can trace their roots back to the now discontinued FreeS/WAN \ac{ipsec} project, an early \ac{ipsec} implementation for Linux. Both support \ac{ike}v1 and \ac{ike}v2, as well as an extensive list of additional features and authentication methods. For this project, we installed both implementations and configured them to use \ac{ike}v1 with \acp{psk} for authentication. The libreswan implementation uses a so-called \emph{ipsec.conf} configuration file to specify connection details including \ac{ike} version, mode and authentication type. The \ac{ipsec} background service is started/restarted via the commands \texttt{ipsec start} or \texttt{ipsec restart}. During learning and fuzzing, the \ac{ipsec} server was restarted before each execution of code, to ensure identical starting conditions.

The configuration file includes a setting to automatically ready the server connection and to wait for incoming connections on startup. On the other hand, strongSwan actually supports two types of configuration files. One more modern one using the Versatile IKE Control Interface (VICI), and another legacy option, also using an \emph{ipsec.conf} configuration file. The \ac{ipsec} service is started using the same commands. To make the configuration file translation between \ac{ipsec} implementations as straightforward as possible, we chose to use the \emph{ipsec.conf} configuration file for both implementations. What follows is an overview of the used \emph{ipsec.conf} settings for both implementations, as well as the full configuration files, shown in listings~\ref{lst:strongswan_config} and \ref{lst:libreswan_config}.

The \emph{ipsec.conf} settings control most facets of an \ac{ipsec} connection. The configuration consists of two major sections, \emph{config} and \emph{conn}. The \emph{config} section contains settings related to the general behavior of the \ac{ipsec} background service that is not limited to an individual connection. In our case, we specify the debugging behavior to log everything in Line~\ref{ln:strongswan_2} and set the \emph{uniqueids} setting to false in Line~\ref{ln:strongswan_3} of both configuration files. The \emph{uniqueids} setting is used to instruct the \ac{ipsec} implementations whether to treat individual IDs as globally unique or not. If set to ``no'', new exchanges with the same ID are applied to the existing connection instance instead of replacing it. We use this option, to ensure that we do not invalidate existing sessions with each subsequent \emph{ISAKMP SA} exchange and to not have the server ignore identical IDs. These settings apply to all connections configured in the configuration file.

\begin{lstlisting}[mathescape=true, float=ht, caption=strongSwan configuration, label=lst:strongswan_config]
	config setup $\label{ln:strongswan_1}$
		charondebug="all" $\label{ln:strongswan_2}$
		uniqueids=no $\label{ln:strongswan_3}$
	
	conn vm1tovm2 $\label{ln:strongswan_4}$
		auto=add $\label{ln:strongswan_5}$
		keyexchange=ikev1 $\label{ln:strongswan_6}$
		authby=secret $\label{ln:strongswan_7}$
		left=10.0.2.1 $\label{ln:strongswan_8}$
		leftsubnet=10.0.2.0/24 $\label{ln:strongswan_9}$
		right=10.0.2.2 $\label{ln:strongswan_10}$
		rightsubnet=10.0.2.0/24 $\label{ln:strongswan_11}$
		ike=aes256-sha1-modp2048! $\label{ln:strongswan_12}$
		esp=aes256-sha1! $\label{ln:strongswan_13}$
		ikelifetime=28800s $\label{ln:strongswan_14}$
		dpdaction=none $\label{ln:strongswan_15}$
		keyingtries=%forever $\label{ln:strongswan_16}$
\end{lstlisting}

Connection specific settings are configured in a \emph{conn} block and are given a name as an identifier, as seen in Line~\ref{ln:strongswan_4}. The \emph{auto} setting in Line~\ref{ln:strongswan_5} sets the default behavior when starting the \ac{ipsec} service. Setting it to ``add'' instructs the connection to be readied and to wait for incoming messages. This option allows us to bring the \ac{ipsec} server into a clean starting state by using either the \texttt{ipsec start} or \texttt{ipsec restart} console commands. Using the \emph{keyexchange} or 
\emph{ikev2} directives respectively in Line~\ref{ln:strongswan_6}, we specify that the communication will use the \ac{ike}v1 protocol. The \emph{authby} setting in Line~\ref{ln:strongswan_7} specifies the use of \acp{psk} for authentication. The next four following lines specify the involved IP ranges and subnets. The server or local IP is always referred to as the left side and the external connection partner is referred to as the right side of the \ac{ipsec} connection. Line~\ref{ln:strongswan_12} specifies the encryption/authentication algorithm to be used for the phase one (\ac{isakmp}) connection with the \emph{ike} keyword. The configured options are read as ``cipher-hashing algorithm-modgroup''. So in our case, the connections are configured to use \SI{256}{\bit} AES-CBC mode for encryption, SHA-1 as a hashing algorithm and the modp2048 Diffie-Hellman Group for key exchanges. Next follows the identical configuration for phase two (\ac{ipsec}) in Line~\ref{ln:strongswan_13}, with the \emph{esp} keyword. We use the same parameters here, as for the prior configuration, however, as keying information for phase two is generated in phase one, we no longer need the modgroup value. The \emph{ikelifetime} keyword in Line~\ref{ln:strongswan_14} determines how long the phase one connection will stay valid before the keys have to be replaced as a brute-force protection. ``dpdaction'' for strongSwan and ``dpddelay'' in combination with ``dpdtimeout'' for libreswan both serve to disable the \ac{dpd} feature of the \ac{ipsec} implementations. \ac{dpd} is a feature that allows \ac{ipsec} servers to probe the status of their connection peers in order to ensure their availability. As this creates additional traffic, we disable this feature for the testing environment. Finally, in Line~\ref{ln:strongswan_16} we allow unlimited keying attempts in order to allow for the efficient fuzzing of phase one messages.

\begin{lstlisting}[mathescape=true, float=ht, caption=libreswan configuration, label=lst:libreswan_config]
	config setup
		plutodebug=all
		uniqueids=no
	
	conn vm1tovm2
		auto=add
		ikev2=no
		authby=secret
		left=10.0.2.1 
		leftsubnet=10.0.2.0/24
		right=10.0.2.2
		rightsubnet=10.0.2.0/24
		ike=aes256-sha1-modp2048
		esp=aes256-sha1
		ikelifetime=28800s
		dpddelay=0
		dpdtimeout=0
		keyingtries=%forever
\end{lstlisting}

As can be seen in listings~\ref{lst:strongswan_config} and ~\ref{lst:libreswan_config}, the two configuration files only differ in a few lines. The main differences are in the debugging setting, as the two implementations use different management backends, and in how \ac{dpd} is disabled. Here, strongSwan has an explicit keyword to disable the function, while libreswan implicitly disables it if related settings are set to 0. Options not specified in the configuration file are set to their default values. Important default values relevant to the thesis include ``aggressive'', ``tunnel'' and ``pfs''. The ``aggressive'' keyword determines if \ac{ike} should run in \emph{Aggressive} mode or \emph{Main} mode, defaulting to Main mode. As Main mode is the more commonly used, as well as the more complicated option (including encryption), we leave this setting at its default value. The other two settings determine the type of \ac{vpn} connection to establish (defaults to tunnel mode) and the use of \ac{pfs} (defaults to \ac{pfs} enabled). Both are left at their default values. Starting the \ac{ipsec} server with the presented configuration options results in an easily-testable, basic \ac{vpn} setup.


\section{Debugging}
During any software development process, roughly 35-50 percent of development time is on the testing and validation of the software~\cite{britton2013reversible}. The development of our custom mapper and fuzzer required a very large amount of debugging, leaning more towards the 50 percent side of the aforementioned statistic. A common scenario was, that the mapper class would do some cryptographic calculations, but the server would return an error. Setting the debugging options in the respective configuration files to ``all'', has the \ac{ipsec} implementations log all internal procedures in great detail. This often gave more insight as to why specific packets were being rejected. However, in certain cases, a manual comparison of strongSwan-generated and our custom-generated \ac{ipsec} packets had to be performed on a byte-by-byte level. The open-source packet sniffing tool Wireshark~\cite{doc:wireshark} was used for this purpose. It allowed us to first connect the strongSwan client and then our own mapper class, all the while recording the traffic. This aided greatly in finding packet-level differences between the two implementations, allowing us to find and fix several bugs in ours. Unfortunately for debugging, \ac{ipsec}, or \ac{ike} to be more specific, encrypts large portions of its communication (everything past the first key exchange). This of course made analyzing the packets impossible, as the relevant information would be encrypted. Luckily, strongSwan allows for the logging of encryption keys and connection cookies. However, this setting is disabled by default. To enable it, one has to edit a different configuration file, namely the \emph{strongswan.conf} file, found usually in the same folder as the \emph{ipsec.conf} file. This other configuration file controls \ac{ipsec} management backend specific options, including a logging level. Setting this level to the highest (four), causes cryptographic keys to also be logged. An excerpt of the relevant log file can be seen in Figure~\ref{fig:keymateriallog}, where one can see the shared Diffie-Hellman secret, as well as all relevant keying material. The actual key used for the encryption of messages is denoted as the \emph{encryption key Ka}. The encryption key is generated by concatenating two hashes of the \emph{SKEYID\_e} and trimming to \SI{32}{\byte} (for AES-CBC-256 encryption). 

\begin{figure}
	\centering
	\includegraphics[width=0.7\linewidth]{images/key_material_log}
	\caption{\ac{ipsec} log excerpt showing keying information.}
	\label{fig:keymateriallog}
\end{figure}


Using the encryption key as well as the cookie of an \ac{ike} connection, the packets can be decrypted in Wireshark, by inputting the values in the corresponding protocol options field, as shown in Figure~\ref{fig:wiresharkdecryption}. Similar cryptographic information can be found in libreswan using the \texttt{ip xfrm state} command. The use of Wireshark for packet-level debugging greatly aided in the development of the custom mapper especially, as oftentimes, the relevant RFC specifications were rather unclear / left open to interpretation. A Wireshark packet dump of a sample \ac{ipsec} connection establishment between two strongSwan participants, as well as the corresponding decryption key and cookies is provided as supplementary material \TODO{supl}. Fortunately, all the effort invested into debugging our implementation resulted in a very detailed logging and testing toolkit for all parts of the implemented protocol. Additionally, once completed, the mapper class greatly reduced the time needed for further testing and fuzzing of the \ac{ipsec} servers, due to their flexible implementation.

\begin{figure}
	\centering
	\includegraphics[width=\linewidth]{images/wireshark_decryption}
	\caption{Decrypting \ac{ipsec} packets in Wireshark.}
	\label{fig:wiresharkdecryption}
\end{figure}

			% setup of VMs, network, VPN etc.
	
	\cleardoublepage
	%----------------------------------------------------------------
%
%  File    :  vpn_learning.tex
%
%  Author  :  Keith Andrews, IICM, TU Graz, Austria
% 
%  Created :  22 Feb 96
% 
%  Changed :  19 Feb 2004
% 
%----------------------------------------------------------------

\chapter{Learning}

\label{chap:Learning}

test		% Automata Learning setup and overview described as in shortpaper, then design and implementation, 
	
	\cleardoublepage			% setup of Fuzzing, then implementation  then maaaaybe Fingerprinting and then implementation
	%----------------------------------------------------------------
%
%  File    :  vpn_fuzzing.tex
%
%  Author  :  Benjamin Wunderling, TU Graz, Austria
% 
%  Created :  22 Feb 96
% 
%  Changed :  19 Feb 2004
% 
%----------------------------------------------------------------

\chapter{Fuzzing}

\label{chap:Fuzzing}
test
	
	\cleardoublepage
	%----------------------------------------------------------------
%
%  File    :  vpn_evaluation.tex
%
%  Author  :  Keith Andrews, IICM, TU Graz, Austria
% 
%  Created :  22 Feb 96
% 
%  Changed :  19 Feb 2004
% 
%----------------------------------------------------------------

\chapter{Evaluation}

\label{chap:Evaluation}
test
\section{Learning Results} \label{subsec:learnresults}
% section where we show and analyze reference and no filter models including model and statistics
TODO: show and analyze Reference and NoFilter model versions --> Maybe move them here from previous chapter?

\begin{figure}[h]
	\centering
	\includegraphics[width=0.7\linewidth]{images/NoFilterA}
	\caption{Most common learned model.}
	\label{fig:nofiltera}
\end{figure}

\begin{figure}[h]
	\centering
	\includegraphics[width=0.9\linewidth]{images/Reference}
	\caption{Clean model learned using retransmission filtering}
	\label{fig:reference}
\end{figure}


% Error model
TODO: analyze error model

\begin{figure}
	\centering
	\includegraphics[width=\linewidth]{images/WithFilterWithErrors}
	\caption{Model with malformed messages}
	\label{fig:withfilterwitherrors}
\end{figure}


% section copmaring KV and Lstar
TODO: Compare KV and Lstar

% section with discovered bug
TODO: Expand on this
faulty library fixes --> a persistent bug that even after several weeks of debugging. Initially we believed this to be caused by non-deterministic behavior of the SUL or problems in our code, but after comparing logs and packet captures, still occasionally would exhibit non-deterministic behavior and therefore crash. Bug appeared to occur randomly at different points during the learning process. Additionally, it did not occur consistently each learning attempt, which made debugging even more difficult. Finally we discovered a very niche bug in a used Diffie-Hellman python library where if the most significant byte was a zero, it would be omitted from the response, causing the local result to be different than the values calculated by the server. As this would only occur in the rare case where the MSB of the DH exchange was zero, this explains the random and difficult to reproduce behavior of the bug. As the library is not a very widespread one, the impact of this bug is presumably not all that high, still it might compromise the security of affected systems and the maintainer has been notified of the problem.

No way this bug would have been found with out the exhaustive testing done by the model learning procedure and without seeing the slight differences in the resulting models that did not crash during the learning process.
\section{Fuzzing Results} \label{subsec:fuzzresults}
test
% Two main parts, eval of learning and eval of testing		% Evaluating the models: Looot of pages showing models of various learned systems etc
	
	\cleardoublepage
	%----------------------------------------------------------------
%
%  File    :  vpn_conclusion.tex
%
%  Author  :  Keith Andrews, IICM, TU Graz, Austria
% 
%  Created :  22 Feb 96
% 
%  Changed :  19 Feb 2004
% 
%----------------------------------------------------------------

\chapter{Conclusion}

\label{chap:Conclusion}


test		% Conclusion
	
	\appendix
	
	\cleardoublepage
	\input{appendix}         % Appendix
	
	
	
	\backmatter
	
	% Use \nocite to ensure that certain references are listed
	% in the bibliography, even if they are not cited in the text.
	% \nocite{KeithMastersThesis}
	% \nocite{KeithPhdThesis}
	
	
	\cleardoublepage
	% for now, switch to language english
	% hack to force unix date for biblio, biblatex 3.11
	\begin{otherlanguage}{english}
		\printbibliography[heading=bibintoc]
	\end{otherlanguage}
	
	
	
	% \cleardoublepage
	% \input{glossary}      % Glossary
	
	
\end{document}

