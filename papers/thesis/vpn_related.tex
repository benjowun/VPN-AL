%----------------------------------------------------------------
%
%  File    :  vpn_related.tex
%
%  Author  :  Keith Andrews, IICM, TU Graz, Austria
% 
%  Created :  22 Feb 96
% 
%  Changed :  19 Feb 2004
% 
%----------------------------------------------------------------

\chapter{Related Work}

\label{chap:Related}
% TODO: check if Feldman is correct
% TODO: add a summary of mentioned related work with each citation
The aim of this chapter is to give a brief overview of related work, focusing mainly on automata learning and testing of secure communication protocols.

Model learning is a popular tool for creating behavioral models of network and communication protocols. The learned models showcase the behavior of the \ac{sul} and can be analyzed to find differences between implementation and specification. Furthermore, the learned models can be used for additional security testing techniques, e.g., for model checking or model-based testing. 

Model learning has been applied to a variety of different protocols, including many security-critical ones. De Ruiter and Poll~\cite{DBLP:conf/uss/RuiterP15} automatically and systematically analyzed TLS implementations by using the random inputs sent during the model learning process to test the \ac{sul} for unexpected and dangerous behavior. The unexpected behavior then had to be manually examined for impact and exploitability. Tappler et al.~\cite{DBLP:conf/icst/TapplerAB17} similarly analyzed various MQTT broker/server implementations, finding several specification violations and faulty behavior. Furthermore, the 802.11 4-Way Handshake of Wi-Fi was analyzed by Stone et al.~\cite{DBLP:conf/esorics/StoneCR18} using automata learning to test implementations on various routers, finding servery vulnerabilities.
Fiterau and Brostean combined model learning with the related field of model checking, in which an abstract model is checked for specified properties to ensure correctness. In their work they learned and analyzed both TCP~\cite{DBLP:conf/cav/Fiterau-Brostean16} and SSL~\cite{DBLP:conf/spin/Fiterau-Brostean17} implementations, showcasing several implementation deviations from their respective RFC specifications.
The Bluetooth Low Energy (BLE) protocol was investigated by Pferscher and Aichernig~\cite{pferscher2021fingerprinting}. In addition to finding several behavioral differences between BLE devices, they were able to distinguish the individual devices based on the learned models, essentially allowing the identification of hardware, based on the learned model (i.e. fingerprinting).

Specifically within the domain of \acp{vpn}, Novickis et al.~\cite{novickis2016protocol} and Daniel et al.~\cite{daniel2018inferring} learned models of the OpenVPN protocol and showcased how to use the learned models to perform protocol fuzzing. In contrast to our approach, they chose to learn a more abstract model of the entire OpenVPN session, where details about the key exchange were abstracted in the learned model. 
Even more closely related to our work, Guo et al. \cite{guo2019model} used automata learning to learn and test the \ac{ipsec}-\ac{ike}v2 protocol setup to use certificate-based authentication. They used the LearnLib~\cite{software:learnlib} library for automata learning and performed model checking of the protocol, using the learned state machine. In contrast, our work focuses on the \ac{ipsec}-\ac{ike}v1 protocol, the predecessor of \ac{ipsec}-\ac{ike}v2, which has not yet been tested with methods utilizing automata learning. \Acp{psk} are used for authentication and the learned model serves as a basis for fuzz-testing. The two protocol versions differ greatly on a packet level, with \ac{ike}v1 needing more than twice the amount of packets to establish a connection than \ac{ike}v2 and also being far more complex to set up.  Guo et al. highlight the complexity of \ac{ike}v1 repeatedly in their work, which emphasizes the need to also test the older version of the protocol as well, especially seeing as it is still in widespread use today~\cite{avm2022}. Our work completes the coverage of learning-based testing approaches for both \ac{ike} versions.

\TODO{Add section on model-based fuzzing, see andrea notes for papers}