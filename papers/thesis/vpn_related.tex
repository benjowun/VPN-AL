%----------------------------------------------------------------
%
%  File    :  vpn_related.tex
%
%  Author  :  Keith Andrews, IICM, TU Graz, Austria
% 
%  Created :  22 Feb 96
% 
%  Changed :  19 Feb 2004
% 
%----------------------------------------------------------------

\chapter{Related}

\label{chap:Related}
% TODO: check if Feldman is correct
% TODO: add a summary of mentioned related work with each citation, total length should be about 2 pages
The aim of this chapter is to give a brief overview of related work, focusing mainly on automata learning and testing of secure communication protocols.
1987 
The concept of learning through the means of membership and equivalence queries was introduced in 1987 by Angluin \parencite{angluin1987learning}. Angluin presented an algorithm for learning regular languages from queries and counterexamples, called $L^*$. In it, a student questions a teacher and constructs a hypothesis based on its responses. The hypothesis is then tested through equivalence queries which check if the hypothesis correctly matches the regular language being learned. While the L* algorithm was originally designed to learn deterministic finite automata (dfa), it can be simply extended to work for Mealy machines by making use of the similarities between dfa and Mealy machines, as shown by Steffen et al. \parencite{steffen2011}. Over time, many related and improved algorithms were published, such as the one proposed by Rivest and Schapire in 1993 in which homing sequences were used to infer finite automata \parencite{Rivest1993Inference}. Another, more recent algorithm came in the form of a redundancy-free active automata learning approach titled TTT by Isberner et al \parencite{Isberner2014TTT}. In this algorithm, essential data is stored in three tree data structures, stripping away unessential information. %TODO: mention version of algorithm used by AALpy

Model learning network protocols for the purpose of testing is a more recent development, with models of protocols like SSH \parencite{fiteruau2017model}, or TCP \parencite{fiteruau2016combining} being learned and used for model checking. Both Novickis et al.~\parencite{novickis2016protocol} and Daniel et al.~\parencite{daniel2018inferring} learned models of the related OpenVPN protocol and used the learned models to perform protocol fuzzing. In a work by Pferscher and Aichernig \parencite{pferscher2021fingerprinting}, learned models were used to fingerprint Bluetooth Low Energy devices (BLE), showing yet another possible use case of automate learning. Guo et al. \parencite{guo2019model} used automata learning to learn and test the IPsec-IKEv2 protocol. They used the LearnLib~\footnote{https://learnlib.de/} library for automata learning and performed model checking of the protocol, using the learned state machine. In contrast, our work focuses on the IPsec-IKEv1 protocol, the predecessor of IPsec-IKEv2, which, to the best of our knowledge, has not yet been tested with methods utilizing automata learning. The protocols differ greatly on a packet level, with IKEv1 needing more than twice the amount of packets to establish a connection than IKEv2.
Additionally we used the \textsc{AALpy}~\footnote{https://github.com/DES-Lab/AALpy} library for automata learning and focused on fuzzing and fingerprinting as opposed to model checking.
