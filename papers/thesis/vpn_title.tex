%----------------------------------------------------------------
%
%  File    			:  thesis-title.tex
%
%  Author  			: Benjamin Wunderling
%  Template from	: Keith Andrews, IICM, TU Graz, Austria	
% 
%----------------------------------------------------------------


% --- Main Title Page ------------------------------------------------

\iffalse
\vspace*{2cm}


\begin{center}
\begin{spacing}{1.1}
\Huge\sffamily\bfseries
Model Learning and Fuzzing of the IPsec-IKEv1 VPN Protocol\\
\end{spacing}

\vspace{3cm}

% \LARGE \sffamily Draft 0.9

\vspace{3cm}

{\LARGE\sffamily
Benjamin Wunderling
}
\end{center}



% --- English Title Page ------------------------------------------------


% based on:
% https://tu4u.tugraz.at/fileadmin/Studierende_und_Bedienstete/Formulare/Diplomarbeit_Vorlage.pdf



\cleardoublepage
\fi

\vspace*{-3cm}

\begin{center}
\includegraphics[height=1cm]{diagrams/tugraz-logo.pdf}

\vspace{3cm}

{\Large\sffamily Benjamin Wunderling, BSc}

\vspace{1cm}

\begin{spacing}{1.1}
\huge\sffamily\bfseries
Model Learning and Fuzzing of the IPsec-IKEv1 VPN Protocol\\
\end{spacing}

\vspace{3cm}


{\Large\sffamily\bfseries Master's Thesis}

\vspace{5mm}

{\small\sffamily to achieve the university degree of}

\vspace{5mm}

{\normalsize\sffamily Diplom-Ingenieur}

\vspace{5mm}

{\normalsize\sffamily
Master's Degree Programme: Computer Science
}


\vspace{1cm}

{\small\sffamily submitted to}

\vspace{5mm}

{\large\sffamily\bfseries Graz University of Technology}



\vspace{1cm}

{\small\sffamily Supervisor}

\vspace{5mm}

{\normalsize\sffamily
Ao.Univ.-Prof.\ Dipl-Ing.\ Dr.\ Bernhard K.~Aichernig \\
Institute of Software Technology (IST)
}


\vspace{1cm}

{\normalsize\sffamily Graz, \thisdate}



\vfill


\end{center}





% --- Pledge ----------------------------------------------------

\cleardoublepage

\vspace*{2cm}


% adapted from:
% https://tu4u.tugraz.at/fileadmin/Studierende_und_Bedienstete/Formulare/Diplomarbeit_Vorlage.pdf
% https://tu4u.tugraz.at/fileadmin/Studierende_und_Bedienstete/Forms/Diploma_thesis_template.pdf
% Email vom 2. Sept. 2015

% and

% Beschluss der Curricula-Kommission für Bachelor-,
% Master- und Diplomstudien vom 10.11.2008
% Genehmigung des Senates am 1.12.2008



\subsection*{Statutory Declaration}
\noindent
\textit{
I declare that I have authored this thesis independently, that I have
not used other than the declared sources / resources, and that I have
explicitly indicated all material which has been quoted either
literally or by content from the sources used. The document uploaded
to TUGRAZonline is identical to the present thesis.}

\vspace{1cm}


\begin{otherlanguage}{austrian}

\subsection*{Eidesstattliche Erklärung}
\noindent
\textit{
Ich erkläre an Eides statt, dass ich die vorliegende Arbeit
selbstständig verfasst, andere als die angegebenen Quellen/Hilfsmittel
nicht benutzt, und die den benutzten Quellen wörtlich und inhaltlich
entnommenen Stellen als solche kenntlich gemacht habe. Das in
TUGRAZonline hochgeladene Dokument ist mit der vorliegenden
Arbeit identisch.}

\vspace{2cm}

\noindent
\parbox[top]{4cm}{
\begin{center}
\underline{\hspace*{4cm}} \\
Date/Datum
\end{center}
}
%
\hfill
%
\parbox[top]{6cm}{
\begin{center}
\underline{\hspace*{6cm}} \\
Signature/Unterschrift
\end{center}
} 

\end{otherlanguage}




% --- English Abstract ----------------------------------------------------


\cleardoublepage

\vspace*{2cm}

\begin{center}
{\Large\sffamily\bfseries Abstract}
\end{center}
% Motivation
In light of the growing number of individuals working from home, the importance of secure \ac{vpn} communication protocols has increased significantly. Consequently, the need for rigorous security testing of \ac{vpn} software has become more critical than ever before. 
In this thesis, we employ \ac{aal} to demonstrate how behavioral models of two black-box \ac{ipsec} \ac{vpn} implementations can be automatically learned. We subsequently utilize the learned models to facilitate security testing the \ac{ipsec} implementations through model-based fuzzing. 
More specifically, we first learn behavioral models of strongSwan and libreswan \ac{ipsec} servers using both the $L^*$ and $KV$ model learning algorithms. Learned models of both \ac{ipsec} implementations are presented and the utilized learning algorithms are compared and evaluated. The learned models are then used to perform model-based fuzzing of the respective implementations. New states are detected by comparing against the previously-learned models. Additionally, we employ search-based methods, including a genetic algorithm-based approach, aided by the learned models, in order to enhance the effectiveness of fuzzing. Our fuzzing analysis uncovered several new security issues, including several standard violations and a potential deadlock state. Furthermore, our model learning approach led to the discovery of a bug in a cryptographic Python library.

