% This is samplepaper.tex, a sample chapter demonstrating the
% LLNCS macro package for Springer Computer Science proceedings;
% Version 2.21 of 2022/01/12
%
\documentclass[runningheads]{llncs}
%
\usepackage[T1]{fontenc}
% T1 fonts will be used to generate the final print and online PDFs,
% so please use T1 fonts in your manuscript whenever possible.
% Other font encondings may result in incorrect characters.
%
\usepackage{graphicx}
% Used for displaying a sample figure. If possible, figure files should
% be included in EPS format.
%
% If you use the hyperref package, please uncomment the following two lines
% to display URLs in blue roman font according to Springer's eBook style:
%\usepackage{color}
%\renewcommand\UrlFont{\color{blue}\rmfamily}
%
\begin{document}
%
\title{Active Automata Learning of an IPsec IKEv1 Server using AALpy}
%
%\titlerunning{Abbreviated paper title}
% If the paper title is too long for the running head, you can set
% an abbreviated paper title here
%
\author{Benjamin Wunderling\inst{1}\orcidID{0000-1111-2222-3333}}

%
\authorrunning{Benjamin Wunderling}
% First names are abbreviated in the running head.
% If there are more than two authors, 'et al.' is used.
%
\institute{TU Graz IST, Graz 8010, Austria \\
\email{benjamin.wunderling@student.tugraz.at}}
%
\maketitle              % typeset the header of the contribution
%
\begin{abstract}
Virtual Private Network (VPN) protocols are widely used to create a secure mode of communication between several parties over an insecure channel. A common use-case for VPNs is to secure access to company networks. Therefore errors in VPN software are often severe. IPsec is a VPN protocol that uses the Internet Key Exchange protocol (IKE). IKE has two versions, IKEv1 and the newer IKEv2. While several papers have investigated IPsec-IKEv2 in the context of Automata learning, no such work has been performed for IPsec-IKEv1. This short paper describes the IPsec-IKEv1 protocol and show the steps taken to learn the state-machine of an IPsec server. We present a learned model and discuss its potential applications for model-based fuzzing and fingerprinting of IPsec implementations.

\keywords{IPsec  \and Automata Learning \and AALpy.}
\end{abstract}
%
%
%
\section{Introduction}
% background
VPNs are used to allow secure communication over an insecure channel. The importance of VPN software has increased dramatically since the beginning of the COVID-19 pandemic due to the influx of people working from home \cite{abhijith2020impact}. This makes finding vulnerabilities in VPN software more critical than ever. IPsec is a VPN protocol and most commonly uses the IKE protocol to share authenticated keying material between involved parties. Therefore, IKE and IPsec are sometimes used interchangeably. We will stick to the official nomenclature of using IPsec for the full protocol and IKE for the key exchange only. IKE has two versions, IKEv1 IKEv2, with IKEv2 being the newer and recommended version \cite{nist791491}. However, despite IKEv2 supposedly replacing its predecessor, IKEv1 is still in widespread use today. This is reflected by the company AVM to this day only offering IKEv1 support for their popular Fritzbox routers \cite{avm2022}.

%Problem
State models of protocol implementations are useful tools in testing. They can for example be used to detect software versions \cite{pferscher2021fingerprinting}, or generate test cases automatically \cite{pferscher2022fuzzing}. One method of generating such models is to use Active Automata Learning. A notable example of an Active Automata Learning algorithm is the L* algorithm by Angluin \cite{angluin1987learning}. In L*, a teacher queries the System under Learning (SUL) and through its responses can construct an automaton describing the behavior of the SUL. This automaton can then be compared with the SUL, adapting it if they show different behaviors. Several papers have investigated IPsec-IKEv2 using Automata Learning, however so far none have looked at IKEv1. 

% My solution
We show the process of learning a state model from an example IPsec-IKEv1 server. We use the Active Automata Learning framework AALpy \cite{muvskardin2022aalpy} for L* Automata Learning, with a custom python interface between AALpy and the IPsec server.

% Structure of Thesis
In this short paper we first go over preliminary information on VPNs and Automata Learning in chapter \ref{chap:2}. In \ref{chap:3} we discuss other related work. In chapter \ref{chap:4} we briefly introduce AALpy and our learning setup. Then we will present our custom interface between AALpy an the IPsec server, discussing design choices and implementation difficulties. Finally we present the learned model and discuss its potential applications and further work in chapters \ref{chap:5} and \ref{chap:6}.



\section{Preliminaries} \label{chap:2}
\subsection{Automata Learning}
% More on Automata Learning, in particular L*
\subsection{IPsec}
% More on VPNs, in particular about IPsec

\section{Related Work} \label{chap:3}%maybe combine
% Discuss related papers (chinese, other similar protocols?)

\section{Learning IPsec} \label{chap:4} %or other title, eihter way, decribe design decisions here
\subsection{Setup}
% describe VMs, IPsec server software, configuration etc
\subsection{AALpy}
% describe aalpy, its functionality we used and limitations
\subsection{Our Interface}
% describe the motivation --> a fuzzable mapper
% describe key code sections and design choices (key management, etc)
% describe problems and solutions --> scapy having very limited packets, network packet level debugging, bad ()low-level) documentation (for strongswan), unclear RFC specs (with regards to actual implementations --> the whole encryption stuff). Retransmits.

\section{Evaluation} \label{chap:5}
% show the learned model, discuss number of states and check if it matches expected behavior
\section{Conclusion} \label{chap:6}
% recap and further planned work

\subsection{A Subsection Sample}
Please note that the first paragraph of a section or subsection is
not indented. The first paragraph that follows a table, figure,
equation etc. does not need an indent, either.

Subsequent paragraphs, however, are indented.

\subsubsection{Sample Heading (Third Level)} Only two levels of
headings should be numbered. Lower level headings remain unnumbered;
they are formatted as run-in headings.

\paragraph{Sample Heading (Fourth Level)}
The contribution should contain no more than four levels of
headings. Table~\ref{tab1} gives a summary of all heading levels.

\begin{table}
\caption{Table captions should be placed above the
tables.}\label{tab1}
\begin{tabular}{|l|l|l|}
\hline
Heading level &  Example & Font size and style\\
\hline
Title (centered) &  {\Large\bfseries Lecture Notes} & 14 point, bold\\
1st-level heading &  {\large\bfseries 1 Introduction} & 12 point, bold\\
2nd-level heading & {\bfseries 2.1 Printing Area} & 10 point, bold\\
3rd-level heading & {\bfseries Run-in Heading in Bold.} Text follows & 10 point, bold\\
4th-level heading & {\itshape Lowest Level Heading.} Text follows & 10 point, italic\\
\hline
\end{tabular}
\end{table}


\noindent Displayed equations are centered and set on a separate
line.
\begin{equation}
x + y = z
\end{equation}
Please try to avoid rasterized images for line-art diagrams and
schemas. Whenever possible, use vector graphics instead (see
Fig.~\ref{fig1}).

% \begin{figure}
% \includegraphics[width=\textwidth]{fig1.eps}
% \caption{A figure caption is always placed below the illustration.
% Please note that short captions are centered, while long ones are
% justified by the macro package automatically.} \label{fig1}
% \end{figure}

\begin{theorem}
This is a sample theorem. The run-in heading is set in bold, while
the following text appears in italics. Definitions, lemmas,
propositions, and corollaries are styled the same way.
\end{theorem}
%
% the environments 'definition', 'lemma', 'proposition', 'corollary',
% 'remark', and 'example' are defined in the LLNCS documentclass as well.
%
\begin{proof}
Proofs, examples, and remarks have the initial word in italics,
while the following text appears in normal font.
\end{proof}
For citations of references, we prefer the use of square brackets
and consecutive numbers. Citations using labels or the author/year
convention are also acceptable. The following bibliography provides
a sample reference list with entries for journal
articles~\cite{ref_article1}, an LNCS chapter~\cite{abhijith2020impact}, a
book~\cite{ref_book1}, proceedings without editors~\cite{ref_proc1},
and a homepage~\cite{ref_url1}. Multiple citations are grouped
\cite{ref_article1,ref_lncs1,ref_book1},
\cite{ref_article1,ref_book1,ref_proc1,ref_url1}.

\subsubsection{Acknowledgements} Please place your acknowledgments at
the end of the paper, preceded by an unnumbered run-in heading (i.e.
3rd-level heading).

%
% ---- Bibliography ----
%
% BibTeX users should specify bibliography style 'splncs04'.
% References will then be sorted and formatted in the correct style.
%
\bibliographystyle{splncs04}
\bibliography{bibliography}
\end{document}
